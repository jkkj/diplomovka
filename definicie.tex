\chapter{Základné pojmy a definície}
\thispagestyle{empty}


\section{Graf}
V teórii grafov sa používa viacero rozdielnych definícii grafu, preto v
tejto kapitole uvádzame základné definície pojmov použitých v našej práci.

Graf je obrazová reprezentácia údajov obsahujúca body - vrcholy a čiary,
ktoré body spájajú - hrany. Hrana je úsek spojitej krivky začínajúci a 
končiaci v bode. Hrana vedie medzi vrcholmi v ktorých má konce.
O vrchole, do/z ktorého hrana vedie a tejto hrane, hovoríme, 
že sú navzájom incidentné. Niektoré definície povoľujú aby hrana začínala a 
končila v tom istom vrchole; takúto hranu nazývame slučka. 
Pokiaľ vedie medzi dvoma vrcholmi viac hrán, hovoríme o násobných hranách. 
Striktné definície grafu nepovoľujú násobné hrany a slučky a ani my sa nimi 
nebudeme zaoberať.

Na reprezentáciu vrcholov sa v teórii grafov používajú prvky množiny a 
hrana je v množine 
hrán reprezentovaná dvojicou vrcholov, ktoré spája, teda dvojprvkovou
podmnožinou množiny vrcholov. Ak rozlišujeme začiatočný a koncový vrchol
hrany, hovoríme, že hrana je orientovaná. Takéto hrany potom zvykneme
reprezentovať usporiadanou dvojicou vrcholov.

Nasledujú formálne definície dôležitých pojmov z teórie grafov. Ak chceme
rozlíšiť, ktorý graf máme na mysli, používame onačenie tohto grafu ako dolný
index. Napríklad množina hrán $E$ grafu $G$ sa označuje $E_{G}$.

\begin{defin}
Graf $G$ je usporiadaná dvojica $(V,E)$, kde
$E \subseteq \{u,v | u \in V, v \in V\}$.

Množine $V$ hovoríme množina vrcholov a množine $E$ množina hrán.

Vrcholy označujeme malými písmenami prevažne z konca latinskej abecedy 
(okolo písmena v). Hrany označujeme malými písmenami zo začiatku latinskej 
abecedy (okolo písmena e) alebo ako dvojicu vrcholov, ktoré spája. Keďže sa
zaoberáme neorientovanými grafmi, platí
$(u,v) \equiv (v,u)$.
\end{defin}

\begin{defin}
Hovoríme, že hrana $e \in E$ je incidentná s vrcholom $v \in V$, ak
$\exists  u \in  V : e = (u,v) $
\end{defin}

\begin{defin}
Stupeň vrchola $v$ je počet hrán s ktorými je incidentný; označujeme deg(v).
Teda $deg(v) = |\{e | v \in e\}|$ \footnote{Keďže sme si hranu definovali ako
dvojprvkovú podmnožinu množiny vrcholov, môžeme si dovoliť tento zápis}
\end{defin}
  
\begin{defin}
Sled $S$ v grafe $G$ je postupnosť vrcholov a hrán končiaca a začínajúca vrcholom taká,
že dva jej susedné prvky sú navzájom incidentné.

Čiže $S = v_{0},e_{0},v_{1},e_{1}, ... ,e_{n-1},v_{n}$; kde $\forall i:$ 
$v_{i} \in V, e_{i} \in E$ ; $\{v_{i},v_{i+1}\} = e_{i}$.
\end{defin}

\begin{defin}
Ťah $T$ v grafe $G$ je taký sled $S$ v tomto grafe, v ktorom sa neopakujú hrany.
$T = S : \forall e_{i},e_{j} \in S : e_{i} \equiv e_{j} \Rightarrow i = j$
\end{defin}

\begin{defin}
Hovoríme, že ťah alebo sled sú uzavreté, ak je ich posledný vrchol totožný s
prvým vrcholom.
\end{defin}

\begin{defin}
Cesta $P$ v grafe $G$ je taký ťah $T$ v tomto grafe, že sa v ňom neopakujú
vrcholy. 
\end{defin}

\begin{defin}
Cyklus $C$ v grafe $G$ je taký uzavretý ťah $T$ v tomto grafe, že sa v ňom
neopakujú vrcholy.
\end{defin}

\begin{defin}
Graf je súvislý, ak v ňom existuje sled obsahujúci všetky vrcholy.
\end{defin}

\begin{defin}
Podgraf $H$ grafu $G$ je taký graf, ktorého množina vrcholov $V_{H}$ je
podmnožinou vrcholov $V_{G}$ grafu $G$. Množina hrán grafu $H$ je
podmnožinou tých hrán grafu $G$, ktoré sú incidentné len s vrcholmi z
množiny $V_{H}$.

Indukovaný podgraf je taký graf, ktorého množina hrán obsahuje všetky hrany,
ktorých koncové vrcholy patria do jeho množiny vrcholov.
\end{defin}

\begin{defin}
Most v grafe $G$je taká hrana $e = \{ u,v \}$ v grafe $G$, že ľubovoľný sled 
obsahujúci vrcholy $u$ a $v$ obsahuje hranu $e$.
\end{defin}

\begin{defin}
Hovoríme, že vrcholy $v, u$ grafu $G$ sú susedné vrcholy, ak $\exists e \in
E_{G}: u,v, \in e$.
\end{defin}

\begin{defin}
Komponent $K$ grafu $G$ je taký jeho indukovaný súvislý podgraf, že v grafe 
$G$ nevedie hrana medzi vrcholom v $K$ a žiadnym vrcholom ležiacim vo 
zvyšku grafu $G$.
\end{defin}

\begin{defin}
Kostra súvislého grafu $G$ je taký jeho súvislý podgraf, kde každá hrana je
most.
\end{defin}

\begin{defin}
Produkt kartézskeho súčinu grafov $G$ a $H$ označujeme $G \times H$, pričom
platí:

$V_{G \times H} = V_{G} \times V_{H}$

$E_{G \times H} = E_{G} \times V_{H} \cup E_{H} \times V_{G} $

Koncové vrcholy hrany $(d,v) \in E(G) \times V(H)$ sú vrcholy $(x,v)$ a
$(y,v)$, kde $x,v \in d; d \in E(G)$, jedná sa teda o hranu $\{(x,v), (y,v)\}$.
Pre hranu $(e,u) \in E(H) \times V(G)$
podobne.
\end{defin}


\section{Vybrané triedy grafov}
Niektoré grafy majú spoločné vlastnosti na základe ktorých ich delíme do
množín - tried grafov.
Teraz si predstavíme niekoľko tried grafov potrebných v ďalších častiach
diplomovky. 

\begin{defin}
Cesta dĺžky $n$ je graf, v ktorom existuje cesta na $n$ vrcholoch a neobsahuje
žiadne ďalšie hrany ani vrcholy. Takýto graf označujeme $P_{n}$.
\end{defin}

Podobne definujeme triedu grafov cykly.

\begin{defin}
Cyklus dĺžky $n$ je graf s $n$ vrcholmi, v ktorom existuje cyklus s $n$
vrcholmi obsahujúci všetky hrany tohto grafu. Cyklus dĺžky $n$ označujeme
$C_{n}$.
\end{defin}

\begin{defin}
n-rozmerná kocka $Q_{n}$ je graf, ktorý vznikol ako kartézsky súčin $n$
činiteľov $P_{2}$.

$Q_{n} \equiv \times_{n}^{i=1} (P_{2})_{i}$

Keď označíme vrcholy grafu $P_{2}$ ako 0 a 1. Budú označenia vrcholov $Q_{n}$
zodpovedať bitovým zápisom čísel od 0 po n-1 a hrany povedú medzi dvomi
vrcholmi, pri ktorých sa zodpovedajúce čísla líšia v práve jednom bite.
\end{defin}