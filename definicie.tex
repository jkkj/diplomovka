\chapter{Základné pojmy a defínície}
\thispagestyle{empty}


\section{Graf}
V teórii grafov sa používa viacero rozdielnych definícii grafu, preto v
tejto kapitole uvádzame základné definície pojmov použitých v našej práci.

Graf je obrazová reprezentácia údajov obsahujúca body - vrcholy a čiary,
ktoré body spájajú - hrany. Hrana je úsek spojitej krivky začínajúci a 
končiaci v bode. Hrana vedie medzi vrcholmi v ktorých má konce.
O vrchole, do/z ktorého hrana vedie a tejto hrane, hovoríme, 
že sú navzájom incidentné. Niektoré definície povoľujú aby hrana začínala a 
končila v tom istom vrchole; takúto hranu nazývame slučka. 
Pokiaľ vedie medzi dvoma vrcholmi viac hrán, hovoríme o násobných hranách. 
Striktné definície grafu nepovoľujú násobné hrany a slučky a ani my sa nimi 
nebudeme zaoberať.

Na reprezentáciu vrcholov sa v teórii grafov používajú prvky množiny a 
hrana je v množine 
hrán reprezentovaná dvojicou vrcholov, ktoré spája, teda dvojprvkovou
podmnožinou množiny vrcholov. Ak rozlišujeme začiatočný a koncový vrchol
hrany, hovoríme, že hrana je orientovaná. Takéto hrany potom zvykneme
reprezentovať usporiadanou dvojicou vrcholov.

Nasledujú formálne definície dôležitých pojmov z teórie grafov. Ak chceme
rozlíšiť, ktorý graf máme na mysli, používame onačenie tohot grafu ako dolný
index. Napríklad množina hrán $E$ grafu $G$ sa označuje $E_{G}$.

\begin{defin}
Graf $G$ je usporiadaná dvojica $(V,E)$, kde
$E \subseteq \{u,v | u \in V, v \in V\}$.

Množine $V$ hovoríme množina vrcholov a množine $E$ množina hrán.

Vrcholy označujeme malými písmenami prevažne z konca latinskej abecedy 
(okolo písmena v). Hrany označujeme malými písmenami zo začiatku latinskej 
abecedy (okolo písmena e) alebo ako dvojicu vrcholov, ktoré spája. Keďže sa
zaoberáme neorientovanými grafmi, platí
$(u,v) \equiv (v,u)$.
\end{defin}

\begin{defin}
Hovoríme, že hrana $e \in E$ je incidentná s vrcholom $v \in V$, ak
$\exists  u \in  V : e = (u,v) $
\end{defin}

\begin{defin}
Stupeň vrchola $v$ je počet hrán s ktorými je incidentný; označujeme deg(v).
Teda $deg(v) = |\{e | v \in e\}|$ \footnote{Keďže sme si hranu definovali ako
dvojprvkovú podmnožinu množiny vrcholov, môžeme si dovoliť tento zápis}
\end{defin}
  
\begin{defin}
Sled $S$ v grafe $G$ je postupnosť vrcholov a hrán končiaca a začínajúca vrcholom taká,
že dva jej susedné prvky sú navzájom incidentné.

Čiže $S = v_{0},e_{0},v_{1},e_{1}, ... ,e_{n-1},v_{n}$; kde $\forall i:$ 
$v_{i} \in V, e_{i} \in E$ ; $\{v_{i},v_{i+1}\} = e_{i}$.
\end{defin}

\begin{defin}
Ťah $T$ v grafe $G$ je taký sled $S$ v tomto grafe, v ktorom sa neopakujú hrany.
$T = S : \forall e_{i},e_{j} \in S : e_{i} \equiv e_{j} \Rightarrow i = j$
\end{defin}

\begin{defin}
Hovoríme, že ťah alebo sled sú uzavreté, ak je ich posledný vrchol totožný s
prvým vrcholom.
\end{defin}

\begin{defin}
Cesta $P$ v grafe $G$ je taký ťah $T$ v tomto grafe, že sa v ňom neopakujú
vrcholy. 
\end{defin}

\begin{defin}
Cyklus $C$ v grafe $G$ je taký uzavretý ťah $T$ v tomto grafe, že sa v ňom
neopakujú vrcholy.
\end{defin}

\begin{defin}
Graf je súvislý, ak v ňom existuje sled obsahujúci všetky vrcholy.
\end{defin}

\begin{defin}
Podgraf $H$ grafu $G$ je taký graf, ktorého množina vrcholov $V_{H}$ je
podmnožinou vrcholov $V_{G}$ grafu $G$. Množina hrán grafu $H$ je
podmnožinou tých hrán grafu $G$, ktoré sú incidentné len s vrcholmi z
množiny $V_{H}$.

Indukovaný podgraf je taký graf, ktorého množina hrán obsahuje všetky hrany,
ktorých koncové vrcholy patria do jeho množiny vrcholov.
\end{defin}

\begin{defin}
Most v grafe $G$je taká hrana $e = \{ u,v \}$ v grafe $G$, že ľubovoľný sled 
obsahujúci vrcholy $u$ a $v$ obsahuje hranu $e$.
\end{defin}

\begin{defin}
Hovoríme, že vrcholy $v, u$ grafu $G$ sú susedné vrcholy, ak $\exists e \in
E_{G}: u,v, \in e$.
\end{defin}

\begin{defin}
Komponent $K$ grafu $G$ je taký jeho indukovaný súvislý podgraf, že v gafe 
$G$ nevedie hrana medzi vrcholom v $K$ a žiadnym vrcholom ležiacim vo 
zvyšku grafu $G$.
\end{defin}

\begin{defin}
Kostra súvislého grafu $G$ je taký jeho súvislý podgraf, kde každá hrana je
most.
\end{defin}

\begin{defin}
Produkt kartézskeho súčinu grafov $G$ a $H$ označujeme $G \times H$, pričom
platí:

$V_{G \times H} = V_{G} \times V_{H}$

$E_{G \times H} = E_{G} \times V_{H} \cup E_{H} \times V_{G} $

Koncové vrcholy hrany $(d,v) \in E(G) \times V(H)$ sú vrcholy $(x,v)$ a
$(y,v)$, kde $x,v \in d; d \in E(G)$. Pre hranu $(e,u) \in E(H) \times V(G)$
podobne.
\end{defin}

\iffalse
\subsection{Značenia v grafe}
Ďalej si zavedieme rôzne značenia - zobrazenia do množín značiek. Značenia používajú najmä na reprezentáciu rôznych vlastností prvku reprezentovaného daným vrcholom grafu alebo vzťahu reprezentovaného hranou. V našej práci budeme značiť najmä konce hrán.\\

\begin{defin}
Značenie f (mapping) je zobrazenie z množiny objektov O (zobrazovaná množina) do množiny značiek Z.\\
$f: O -> Z ; \forall o \in O  \exists z \in Z  : f(o)=z$
\end{defin}

\begin{defin}
Značenie konca hrán je značenie kde zobrazovanú množinu tvoria dvojice pozostávajúce z vrcholu a s ním incidentnej hrany.
\end{defin}

\section{Vybrané triedy grafov}
Následne si zadefinujeme triedy grafov, ktorými sa budeme bližšie zaoberať a ďalšie potrebné pojmy z teórie grafov použité v tejto práci. 

\subsection{Úplný graf}
\begin{defin}
Úplný graf G = (V,E) je graf v ktorom pre každú dvojicu rôznych vrcholov existuje hrana s ktorou sú oba incidentné.\\
tj. $\forall u,v \in V\ \exists e \in  E :\ e=(u,v)$, kde $u \neq v$\\
ozn. $G \equiv K_n;\ n=|V|$
\end{defin}
\subsection{N-partitný graf}
\begin{defin}
N-partitný graf G = (V,E) je graf v ktorom sa dá množina vrcholov V rozdeliť na n neprázdnych disjunktných podmnožín - partícií, ktoré ju pokrývajú a zároveň pre každú z nich platí, že neobsahuje dvojicu vrcholov incidentných s rovnakou hranou, teda ľubovoľná hrana je incidentná najviac s jedným vrcholom tejto množiny.\\
ozn. Pre 2-partitný graf sa zaužívalo pomenovanie bipartitný.
\end{defin}

\begin{defin}
Úplný n-partitný graf G = (V,E) je graf v ktorom pre každú dvojicu vrcholov, ktoré nie sú v rovnakej partícii existuje v E hrana s ktorou sú oba incidentné.\\
ozn. $K_{a_1,...a_n}$, kde $a_i$ je počet vrcholov v i-tej partícii
\end{defin}

\subsection{Hyperkocka}
Hyperkocka, inak nazývaná aj n-rozmerná kocka je graf inšpirovaný geometrickým významom svojho názvu. Každému vrcholu v telese zodpovedá vrchol v grafe. Dva vrcholy v grafe sú incidentné s rovnakou hranou vtedy a len vtedy ak medzi vrcholmi v telese, ktoré reprezentujú, vedie hrana.
\begin{defin}
TODO formálne
\end{defin}
\subsection{Mriežka}
Mriežka je graf reprezentujúci diagram mriežku.
\begin{defin}
TODO
\end{defin}
\subsection{Kubický graf}
\begin{defin}
Kubický graf G je graf v ktorom majú všetky vrcholy stupeň tri.
\end{defin}
\section{Agent}
Ďalej si zadefinujeme agenta. Agent je pre nás objekt vykonávajúci určité inštrukcie, majúci určitú výpočtovú silu, určitú pamäť, možnosti akcie a nachádzajúci sa v konkrétnom vrchole grafu. Následne si zadefinujeme agenta s možnosťou meniť vrchol v ktorom sa nachádza a pravidlá pre takúto zmenu.\\
\begin{defin}
Agent je -ica ... TODO tu bude možno len všeobecný pokec v závislosti od toho, či bude vhodné zjednocovať definície agentov
\end{defin}

\begin{defin}
Mobilný agent je agent, ktorý má v množine pravidiel aj pravidlá na pohyb vo vrcholoch.
TODO
\end{defin}

\fi