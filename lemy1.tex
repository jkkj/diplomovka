%\cleardoublepage
\phantomsection

\input mriezka.tex

\chapter{N-rozmerná kocka}
V tejto kapitole si v úvode pripomenieme niekoľko faktov o grafoch - 
hyperkockách dôležitých pre pochopenie zvyšku kapitoly. V ňom sa venujeme
výsledkom výskumu konštrukcie grafov z triedy hyperkociek našim modelom.

Vrcholy spojené hranou sa líšia práve v jednej súradnici. Každej hrane z
vrcholu možno priradiť súradnicu v ktorej sa jej druhý vrchol od tohto
odlišuje. Toto priradenie je vzhľadom na hrany incidentné s jedným vrcholom
 injekcia.

Susedia vrchola sa navzájom líšia v práve dvoch súradniciach. Polohu agenta
môžeme vyjadriť vrcholom, v ktorom sa práve nachádza. Zodpovedá jej teda
binárny reťazec, ktorého hodnoty bitov budeme volať súradnice. Súradnica v
hyperkocke môže mať hodnotu 0 alebo 1. Po dvoch zmenách je teda rovnaká.
 Pri pohybe po hrane mení agent práve jednu súradnicu svojej pozície. 
Z tohto vyplývajú obmedzenia
pre úsporu pohybov zväčšovaním množiny značiek pre porty. 
Napríklad pri použití dvoch
značiek pre porty potrebuje agent na vytvorenie dvoch nových portových hrán
aspoň 5 ťahov. Úspora nastáva, keď port pre druhú hranu otvorí už počas
cesty za otvorením portu prvej hrany.

\section{Všeobecné výsledky}

\begin{veta}
Dva vrcholy, ktoré sa navzájom líšia v práve dvoch súradniciach majú práve dvoch
spoločných susedov.
\end{veta}
\begin{proof}
Susedné vrcholy sa líšia práve v jednej súradnici. Spločný sused dvoch
vrcholov sa líši v práve jednej súradnici od každého z nich. Teda má všeky
súradnice rovnaké ako jeden z nich a líši sa v jedej z tých dvoch. Teda ju
má rovnakú ako druhý vrchol. Druhú z týchto súradníc má potom odlišnú od
druhého vrchola. Čiže sa od každého z nich líši v práve jednej súradnici.
Toto dáva práve dva rôzne vrcholy, ktoré susedia z oboma. Vrchol, ktorý
sa od jedného z vrcholov líšiacich sa v dvoch súradniciach líši v nejakej
inej, sa od druhého líši v troch súradniciah a teda s ním nesusedí.
\end{proof}

Otázkou ostáva, koľko hrán vie agent vytvoriť počas jednej návštevy vrcholu
otváraním portov? 
\begin{veta}
Pri jednej návšteve vrcholu po prejdení sledu dĺžky n vrcholov (vrátane toho
v ktorom sa práve nachádza) môže agent vytvoriť najviac 
$(frac{n-1}{2}) > \#dimenzii$
hrán otvorením portu.
\end{veta}
\begin{pozn}
Rátajú sa hrany, ktoré vzniknú hneď tj. pri ktorých agent otvorením portov
vytvorí druhý (posledný) koniec.
\end{pozn}
\begin{proof}
Je zrejmé: že vo všetkých vrcholoch, kam majú tieto hrany
viesť, už agent bol; medzi dvomi vrcholmi vedie najviac jedna hrana. Z tohto
a predošlej vety vyplýva, že po návšteve n vrcholov môže agent otvorením
portu vytvoriť z vrcholu, kde sa nachádza, najviac $(frac{n-1}{2}) > \#dimenzii$ hrán. Z
týchto $(frac{n-1}{2})$ vrcholov sa od neho každý líši v inej súradnici,
lebo ak by sa dva líšili v rovnakej, museli by to byť totožné vrcholy. 
\end{proof}
\begin{veta}
Existuje taká postupnosť $n$ vrcholov, že pri návšteve posledného agent
vytvorí $(frac{n-1}{2}) > >\#dimenzii$ hrán.
\end{veta}
\begin{proof}
Ak vezmeme takúto postupnosť odzadu, vyzerala by až na substitúciu dimenzií
takto: 1, 2, 1, 3, 2, 4, 3, 5, 4, 6, 5, ...
\end{proof}
\begin{pozn}
Avšak ľahko vidieť, že žiadne iné portové hrany by medzi vrcholmi v tejto
postupnosti vzniknúť nemohli.
\end{pozn}
Skúsme teda nájsť postupnosť (cestu v hyperkocke) s najväčším možný počtom 
portových hrán medzi jej vrcholmi.
\begin{ozn}
O hrane povieme, že vedie v dimenzii k, ak sa vrcholy, ktoré spája, líšia v
práve tejto dimenzii. 
\end{ozn}
Je jasné, že v akomkoľvek súvislom slede vrcholov povedie hrana
medzi dvojicami nasledujúcich vrcholov, ide o hrany po ktorých agent prejde
z vrcholu do vrcholu.
Portová hrana môže vzniknúť medzi vrcholmi, ktoré sa líšia v práve jednej
súradnici, lebo iné hrany v hyperkocke nie sú. 
\begin{ozn}
Poloha agenta je vyjadrená súradnicami v jednotlivých dimenziách kocky. v
každej dimenzii môže nadobúdať jednu z dvoch súradníc. Označme ich $0$ a $1$.
Pri prechode po hrane agent preklopý jednu z týchto súradníc v práve jednej
dimenzii na druhú.
\end{ozn}

\begin{veta}
V ľubovoľnej postupnosti hrán medzi vrcholmi hyperkocky spojenými hranou, sa
nachádza párny počet hrán v každej dimenzii okrem dimenzie v ktorej sa tieto
vrcholy líšia a teda v nej vedie hrana medzi nimi.
\end{veta}
\begin{proof}
Takéto dva vrcholy sa líšia v práve jednej dimenzii. Ak agent zmenil pri 
pohybe svoju polohu v nejakej dimenzii nepárny počet 
krát, tak je v tejto dimenzii v inej polohe ako bol na začiatku. Ak to nie je 
dimenzia v ktorej sa líši koncový vrchol od začiatočného - agent sa
nenachádza v koncovom vrchole, lebo polohy agenta a tohto vrcholu nie sú
rovnaké.
\end{proof}

\begin{ozn}
Ak viem súradnice vrchola, môžem postupnosť ťahov agenta zapísať ako postupnosť 
zmien jeho súradníc v jednotlivých dimenziách. 
\end{ozn}
Ak sa nachádza v takejto postupnosti medzi
vrcholmi, medzi ktorými vzniká portová hrana dvakrát hneď za sebou číslo
nejakej dimenzie, znamená to, že agent šiel tam a späť a tieto dva prvky
môžeme z postupnosti vylúčiť, lebo je to z pohľadu tejto dvojice vrcholov
zbytočný pohyb. Ak takto zredukujeme niektorú postupnosť, získame inú, v
ktorej ide agent po podmnožine hrán z pôvodnej a na rovný alebo menší počet
pohybov.
\begin{pozn}
Pripomeniem, že hľadáme taký sled, že počet portových hrán medzi jeho
vrcholmi je na daný počet značiek pre porty a počet portových hrán, ktoré
vzniknú medzi jeho vrcholmi, najkratší možný.
\end{pozn}

Ak má agent dostatok značiek pre porty, nezaujíma nás poradie v akom vrcholy
postupnosti navštívil. Úlohu teda môžeme zredukovať na hľadanie takej množiny
vrcholov veľkosti m, ktorá tvorí zo všetkých takto veľkých množín najmenší rez
od zvyšku grafu. Pri konštantnom stupni vrcholov to znamená, že v takej vedie 
najviac hrán medzi vrcholmi v tejto množine.
Keď sa pozrieme na vrcholy a pravidlá ich spájania hranami, zistíme, že
tieto vedú len v nejakej kocke a čím viac rozmerov kocka má, tým viac hrán
vedie medzi jej vrcholmi. Teda aj v n-rozmernej kocke povedie najviac hrán
medzi vrcholmi takej množiny, ktorej indukovaný graf je izomorfný kocke.
Vrchol k-rozmernej kocky sa nachádza v k (k-1)-rozmerných kockách, ktoré sú
jej podgrafmi. Teda budeme hľadať množinu m bodov, ktoré sa dajú
pokryť kockami čím vyššieho rozmeru a nemusíme si všímať kocky ktoré sú
podgrafmi týchto kociek. tieto kocky sa však môžu prekrývať.

\begin{veta}
S každou hyperkockou v ktorej vrchol neleží ho spája najviac jedna hrana.
Ktorá vedie v inej dimenzii ako sú dimenzie tejto hyperkocky.
\end{veta}
\begin{proof}
Každý vrchol hyperkocky má hrany vo všetkých jej dimenziách. Vo zvyšných
dimenziách sú vrcholy hyperkocky totožné, lebo sa dva susedné líšia práve v
jednej. Vrchol mimo hyperkocky spojený s vrcholom v nej sa od neho líši v
inej dimenzii a v tejto sa líši teda aj od všetkých ostatných. Zároveň je vo
všetkých dimenziách hyperkocky totožný s vrcholom z nej a teda sa lííši v
aspoň jednej dimenzii hyperkocky od jej ostatných vrcholov a má teda s nimi
aspoň dve rodielne dimenzie. Nemôže ho preto s nimi spájať hrana.
\end{proof}

\begin{veta}
Bijektívna substitúcia dimenzií na všetkých vrcholoch podgrafu hyperkocky 
zachováva
hrany medzi vrcholmi. Rovnako aj preklopenie súradníc v niektorej dimenzii.
\end{veta}
\begin{proof}
Ak sa dva vrcholy líšia práve v jednej dimenzii, budú sa v práve jednej
dimenzii líšiť aj po tejto substitúcii, lebo dimenzie v ktorých boli rovnaké
sa zobrazia na rovnaké a tá v ktorej sa kíšili sa u oboch zobrazí na rovnakú
a budú sa v nej líšiť aj naďalej; teda medzi nimi povedie hrana.
Pri preklopení súradníc sa medzi vrcholmi zachovajú rovnosti a rozdiely v
danej dimenzii, teda aj hrany.
\end{proof}

\begin{veta}
Vrcholy súvislého podgrafu hyperkocky s počtom vrcholov m sa navzájom líšia
v najviac m-1 súradniciach.
\end{veta}

\begin{proof}
Vezmime kostru tohto podgrafu. Táto má m-1 hrán. Každá vedie v niektorej
dimenzii. Cesta v tejto kostre udáva v ktorých dimenziách sa daná dvojica
jej koncových vrcholov líši. Všetky cesty v tejto kostre idú cez najviac m-1
rôznych dimenzií, čo je jej počet hrán. V ostatných dimenziách sa dané
vrcholy nelíšia, majú v nich teda navzájom rovnaké súradnice.
\end{proof}

\begin{veta}
Podmnožina m bodov hyperkocky, ktoré indukujú graf s najvyšším počtom hrán je
súvislá.
\end{veta}

\begin{proof}
Nech existujú dva komponenty. Na jednom z nich urobíme 
postupné popreklápanie súradníc tak, aby susedil niektorý jeho bod s bodom
v druhom komponente a teda medzi nimi viedla hrana. Čiže vyberieme jednu
dimenziu v ktorej sa líšia a vo všetkých ostatných dimenziách, kde majp
rozdielne súradnice preklápame postupne súradnice jedného z vrcholov. 
V prípade, že počas tohot postupu vznikne hrana spájajúca vyhrané dva
komponenty, skončíme skôr. Pôvodné hrany sa
zachovajú a jedna nová pribudne, čo je spor s najvyšším počtom hrán.
Tieto kroky môžeme urobiť bez toho, aby sa dva vrcholy zobrazili na seba.
Ak budeme robiť preklápanie súradníc po jednotlivých dimenziách,
v tom prípade, ak by sa nejaký vrchol transformovaného komponentu
podgrafu zobrazil na niektorý vrchol druhého komponentu, musela by 
medzi nimi viesť predtým hrana, teda stav v ktorom by boli komponenty
prepojené a preklápanie by teda už nepokračovalo.
\end{proof}

Potrebujeme vybrať m vrcholový indukovaný podgraf hyperkocky tak, aby počet 
hrán v ňom bol maximálny možný.

\begin{veta}
Vrcholy podgrafu hyperkocky sa dajú rozdeliť na dve množiny A,B podľa súradnice vo
vybranej dimenzie. Počet hrán vedúcich z vrcholov v jednej množine do
vrcholov v druhej množine je rovný alebo menší ako mohutnosť menšej z nich.
Čiže $|\{e|e = (a,b);a \in A; b \in B\}| <= min\{|A|,|B|\}$
\end{veta}
\begin{proof}
Každý vrchol má v jednej dimenzii najviac jednu hranu a hrana vedie v
danej dimenzii medzi dvomi vrcholmi vtedy ak je to jediná dimenzia v ktorej
majú rozdielne súradnice. V inej dimenzii medzi A a B hrana viesť nemôže,
lebo by musela viesť medzi vrcholmi líšiacimi sa v najmenej dvoch 
dimenziách - v dimenzii hrany a v dimenzii podľa súradníc ktorej 
sú A a B vytvorené. Teda hrany medzi A a B môžu viesť len v jednej dimenzii
a do menšej z A,B môže viesť najviac toľko hrán v jednej dimenzii, koľko má
vrcholov.
\end{proof}

\begin{veta}
\label{izomorfne}
Keď vezmeme z podgrafu hyperkocky len vrcholy, ktoré majú hranu v jednej
vybranej dimenzii a označíme nimi indukovaný podgraf G. Vieme vrcholy G
rozdeliť na dva izomorfné indukované podgrafy hyperkocky $G_{1}$ a $G_{2}$
podľa súradnice vrchola v tejto dimenzii.
\end{veta}

\begin{proof}
Vrcholy, medzi ktoými vedie hrana sa líšia v práve jednej dimenzii, v
ostatných dimenziách sú rovnaké. Keď teda vezmeme hrany vedúce medzi 
$G_{1}$ a $G_{2}$ a vytvoríme podľa nich bijektívne zobrazenie medzi týmito
podgrafmi tak, aby sa dva vrcholy medzi ktorými vedie hrana vo vybranej
dimenzii zobrazili na seba, dostaneme izomorfizmus. Ak totiž viedla medzi
dvomi vrcholmi hrana, bude hrana viesť aj medzi ich obrazmi, lebo tieto majú
rovnaké súradnice ako ich vzory vo všetkých dimenziách okrem jednej, kde sa
oba obrazy od svojich vzorov líšia a teda majú súradnicu v tejto dimenzii 
rovnakú; obaja sú v komponente, kde majú túto súradnicu rovnakú všetky
vrcholy. Ak sa teda líšili v jednej súradnici vzory, budú sa aj obrazy a ak
sa líšili vzory vo viacerých súradniciach, budú sa v rovnakých súradniciach
líšiť aj obrazy. Zobrazenie  je teda izomorfizmus.
\end{proof}

\begin{veta}
\label{delenie}
Existuje indukovaný podgraf $G$ hyperkocky na m vrcholoch, ktorý je možné 
postupne rozdeliť
podľa dimenzií až po jednovrcholové indukované podgrafy tak, že v každom
kroku rozdelíme vrcholy každého podgrafu s počtom vrcholov aspoň dva do
dvoch podmnožín $A$ a $B$. Takých, že $|A| - |B| <= 1$.
\end{veta}

\begin{proof}
Dôkaz bude konštrukčný. Budeme postupne deliť m vrcholovú množinu i ďalšie z
nej vzniknuté množiny na menšie, až kým nedostaneme m jednoprvkových množín
vrcholov. Pritom budeme postupne určovať súradnice jednotlivých vrcholov.

Budeme postupovať postupne po dimenziách, až kým budúd všetky množiny
jednoprvkové. Vezmeme množinu $M$, rozdelíme ju na dve časti $M_{1}, M_{0}$
také že $|M_{1}| - |M_{0}| <= 1$. Vrcholy v $M_{1}$ budú mať v tejto
dimenzii súradnicu 1 a vrcholy v $M_{0}$ budú mať v tejto dimenzii súradnicu
0. Ak je množina jednoprvková a delenie niektorých iných ešte pokračuje,
dostane v každej ďalšej dimenzii súradnicu 1. Ak má kocka viac dimenzií ako
je potrebných, aby obsiahla všetkých m vrcholov podgrafu, tak môžu dostať
tieto vrcholy v týchto zvyšných dimenziách ľubovoľné súradnice, ale pre
všetky vrcholy musí byť súradnica v každej z týchto dimenzií rovnaká.
Tým sme zistili súradnice vrcholov vo všetkých dimenziách, podľa ktorých
podgraf delíme. Pre účely tejto vety na súradniciach ostatných dimenzií
nezáleží a môžu byť ľubovoľné.
\end{proof}

\begin{veta}
\label{mensia}
Pri delení z vety \ref{delenie} vedie medzi množinami vrcholov $M_{1}$ a $M_{0}$,
ktoré vznikli rozdelením jednej množiny M podľa tohto delenia, práve toľko
hrán, ako je mohutnosť menšej z týchto množín. Pokiaľ, ak má kocka viac 
dimenzií ako
je potrebných, aby obsiahla všetkých m vrcholov podgrafu, tak môžu dostať
tieto vrcholy v týchto zvyšných dimenziách ľubovoľné súradnice, ale pre
všetky vrcholy musí byť, na rozdiel od predošlej vety, súradnica
 v každej z týchto dimenzií rovnaká.
\end{veta}

\begin{proof}
Stačí ukázať, že menší z indukovaných podgrafov má vo väčšom izomorfný obraz
podľa vety \ref{izomorfne}. Vtedy vedie z každého vrchola v menšom podgrafe
hrana do nejakého vrchola vo väčšom podgrafe a týchto hrán je presne toľko, koľko je
vrcholov menšieho podgrafu.
Teda potrebujeme dokázať, že vo väčšom podgrafe existuje skupina vrcholov,
ktorej budú súradnice v dimenziách pridelené rovnako ako vrcholom v menšom
podgrafe. Prideľovanie súradníc je deterministické, teda rovnako veľkým
množinám vrcholov v rovnakej fáze budú poprideľované súradnice vo zvyšných
dimenziách rovnako. Na menší a väčší podgraf sa rozdelia grafy s nepárnym
počtom vrcholov.
Použijeme úplnú indukciu. Báza indukcie bude množina veľkosti tri. Menšie
podgrafy sú už len veľkosti dva a jedna. Jednovrcholový sa už ďalej nedelí a
dvojvrcholový sa rozdelí na dva jednovrcholové.
Predpokladáme, že pre delenie množín s mohutnosťou menšou ako n platí veta.
Dokážeme, že potom musí platiť aj pre množinu M mohutnosti n: Ak je n párne,
dá sa množina rozdeliť na dve množiny rovnakej mohutnosti, ktorých vrcholom
sa zvyšné dimenzie poprideľujú rovnako. Teda každý vrchol v jednej bude mať
iný vrchol v druhej množine s ktorým sa bude líšiť v práve jednej dimenzii
a to v tej podľa ktorej sme M rozdelili.
Ak n je nepárne, vzniknú dve množiny $M_{1}$, $M_{2}$. Obe z nich už
podľa IP spĺňajú tvrdenie vety. Rozdelia sa každá na dve množiny, pričom z
množiny s párnou mohutnosťou vzniknú dve rovnako veľké množiny
$M_{11}$,$M_{12}$ a množina $M_{21}$
tejto veľkosti vznikne aj z množiny s nepárnou mohutnosťou. Druhá množina
$M_{22}$ z tejto množiny bude mať o prvok viac alebo menej.
Prvé tri množiny sa už budú deliť rovnako, teda viem povyberať trojice
vrcholov, každý z inej množiny, ktoré budú mať vo zvyšných dimenziách
rovnaké súradnice. Medzi dvomi vrcholmi z $M_{11}$,$M_{12}$, označme ich 
$m_{11}$,$m_{12}$,  povedie hrana,
lebo sa líšia práve v jednej dimenzii. Tretí vrchol, označme ho $m_{21}$
 bude spojený hranou s
niektorý z tých dvoch, pretože pri druhom delení jeden z nich dostal rovnakú
súradnicu ako on v dimenzii druhého delenia a teda sa líšia iba v dimenzii
prvého delenia množiny M. Ak je prvok $m_{21}$ spojený s prvkom z
$M_{22}$, označme ho $m_{22}$, potom je $m_{22}$ spojený aj s tým prvkom
z dvojice $m_{11}$,$m_{12}$, s ktorým nie je spojený $m_{21}$.
Prvok $m_{22}$ neexistuje iba vtedy, ak $|M_{22}|<|M_{21}|$. 
Ak $|M_{22}|>|M_{21}|$ potom v $M_{22}$ majú všetky vrcholy z $M_{21}$
izomorfný obraz a teda aj príslušná časť z párnej množiny.
Z horeuvedeného vyplýva, že menšia množina z prvého delenia má obraz,
izomorfný a spĺňajúci podmienky vety \ref{izomorfne}, 
vo väčšej množine vzniknutej z tohto delenia a keďže 
medzi vzorom a obrazom vedie hrana, veta platí.
\end{proof}

\begin{veta}
\label{maximum}
Maximálny počet hrán, ktoré môžu viesť v podgrafe hyperkocky indukovanom m
bodmi je daný postupnosťou $a_{m}$, kde $a_{1} = 0; a_{2} = 1;$ a pre 
$\forall i < 2; a_{i} = \lfloor \frac {i}{2} \rfloor + a_{\lfloor \frac{i}{2}
\rfloor}
+ a_{\lceil \frac {i}{2} \rceil } $
\end{veta}
\begin{proof}
V dôkaze použijeme úplnú indukciu. Báza indukcie je očividná: hyperkocka
nemá slučky a teda v jej jenovrcholovom indukovanom grafe nevedie 
hrana a hyperkocka nemá
násobné hrany, teda v jej indukovanom dvojvrcholovom podgrafe 
môže viesť najviac jedna hrana. Predpokladáme, že pre všetky členy s
indexom menším ako m platí, že udávajú najväčší možný počet hrán v
indukovanom podgrafe hyperkocky zo všetkých množín jej bodov, ktorých 
mohutnosť je rovná  indexu daného člena postupnosti. ďalej predpokladáme, že
pre každého člena postupnosti s menším indexom platí tvrdenie vety.
Ukážeme, že existuje indukovaný podgraf hyperkocky s m bodmi 
s najvyšším možným počtom hrán G taký, ktorý niektorá dimenzia delí na
polovice $G_{1}$ a $G_{2}$, teda na množiny vrcholov, ktorých mohutnosti 
sa líšia najviac o
jedna. To vyplýva z toho, že neexistuje taký podgraf hyperkocky, 
ktorý by delila inak a mal by zároveň vyšší počet hrán.
Nech taký podgraf $G^{,}$ existuje. Potom ho vybraná dimenzia delí na podgrafy s
počtom bodov $a_{i}$, $a_{j}$. Keďže sú obe neprázdne, sú menšie ako m a teda
pre ne platí indukčný predpoklad. $a_{i} = \lfloor \frac {i}{2} \rfloor + 
a_{\lfloor \frac{i}{2} \rfloor}+ a_{\lceil \frac {i}{2} \rceil }$ a $a_{j} = 
\lfloor \frac {j}{2} \rfloor + 
a_{\lfloor \frac{j}{2} \rfloor}+ a_{\lceil \frac {j}{2} \rceil }$
Bez ujmy na všeobecnosti, nech i < j. Aby sme zjednodušili indexovanie
označme $a_{\lfloor \frac{i}{2} \rfloor} = a_{i1}; a_{\lceil \frac {i}{2} \rceil }
= a_{i2}; a_{\lfloor \frac{j}{2} \rfloor} = a_{j1}$ a $a_{\lceil \frac {j}{2} \rceil }
= a_{j2}$.
Je zrejmé, že $a_{i2} - a_{i1} <= 1$ a $a_{j2} - a_{j1} <= 1$. Ak teda
rozdelíme G podľa dimenzie, ktorá ho delí na $G_{1}$ a $G_{2}$, ich mohutnosti
budú bez ujmy na všeobecnosti $a_{i2} + a_{j1}; a_{j2} + a_{i1}$. Z IP
zároveň platí, že $G_{1}$ a $G_{2}$ majú aspoň tak dobré rozdelenie ako na
množiny veľkostí $a_{i2}, a_{j1}$ respektíve $a_{j2}, a_{i1}$.
Počet hrán, ktoré vedú vo podgrafe $G^{,}$ je teda
$a_{m^{,}}=i + \lfloor \frac{i}{2} \rfloor 
+ \lfloor \frac{j}{2} \rfloor + a_{i1} + a_{i2} + a_{j1} + a_{j2}$.
Je zrejmé, že menšia z množín na ktoré sa rozdelí G bude mať veľkosť
$\lfloor \frac{m}{2} \rfloor$ a v týchto podmnožinách vedie aspoň $a_{i1} +
a_{j2} + \lfloor \frac{i}{2} \rfloor $ respektíve $a_{j1} + a_{i2} + 
\lceil \frac{i}{2} \rceil$ hrán.
Čiže $a_{m} >= \lfloor \frac{m}{2} \rfloor + a_{i1} +
a_{j2} + \lfloor \frac{i}{2} \rfloor + a_{j1} + a_{i2} + 
\lceil \frac{i}{2} \rceil$
Z toho si ľahko ukážeme, že $a_{m} >= a_{m^{,}}$.
 $a_{m} - a_{m^{,}} >= \lfloor \frac{m}{2}\rfloor + \lceil \frac{i}{2}
\rceil - i - \lfloor \frac{j}{2} \rfloor$ a keďže 
$\lfloor \frac{m}{2}\rfloor = \lfloor \frac{i+j}{2}\rfloor >= \lfloor \frac{i}{2} \rfloor + \lfloor \frac{j}{2} \rfloor$
potom platí, že $a_{m} - a_{m^{,}} >=0$. Tým sme dokázali, že neexistuje
lepšie rozdelenie podgrafu hyperkocky indukovaného m vrcholmi ako rozdelenie
na polovice. Z predošlých viet vyplýva, že takéto rozdelenie možné je.
\end{proof}




\begin{veta}
\label{rozdel}
Zoberme jednu dimenziu a rozdeľme body na dve množiny, podľa toho,
či v nej majú 1 alebo 0. Platia dve veci: medzi týmito dvomi množinami vedie
najviac toľko hrán, koľko má menšia z nich vrcholov; v iných dimenziách už
medzi týmito vrcholmi hrany nevedú.
\end{veta}
\begin{proof}
Prvá vec vyplýva z toho, že každý vrchol má v jednej dimenzii najviac jednu
hranu (pri nekompletnej hyperkocke, pri kompletnej práve jednu).
Druhá vyplýva z toho, že hrana spája vrcholy, ktoré sa líšia práve v jednej
súradnici a vedie v dimenzii tejto súradnice. Ak by mala viesť medzi týmito
dvomi množinami hrana v nejakej inej súradnici, musela by spájať vrcholy,
ktoré sa líšia v najmenej dvoch súradniciach (v tej ktorá rozdeľuje vrcholy
na dve množiny a v tej v ktorej vedie táto hrana), čo nemôže.
\end{proof}


\begin{veta}
Majme množinu množín vrcholov, označme
ju M. Nech existuje taká postupnosť spájania týchto množín,
ktorého súčet bude najvyšší možný. Súčtom spájania nazývame najvyšší možný
počet hrán, ktoré môžu viesť medzi mnnožinami vrcholov, ak sú rozdelená po
dimenziách ako vo vete \ref{rozdel}.
Nech sa v ňom nenachádza spojenie
najmenšej množiny s inou v prvom kroku, ukážeme že existuje poradie, kde sa
nachádza o krok skôr, ktoré je aspoň také dobré alebo pôvodné nie je
najlepšie.
\end{veta}
\begin{proof}
Ak sa najmenší prvok spája s množinou, ktorá bola v M, vieme spojenie minima
s ňou presunúť do úplne prvého kroku spájania.
Ak sa spája s množinou, ktorá vznikla spojením dvoch množín, premiestnime
krok pripojenia minima hneď za krok pri ktorom táto množina vznikla. Označme
  množiny z ktorých táto množina vznikla A,B a mohutnosť minima 
označme C, |A| = a; |B| = b; |C| = c.
Bez ujmy na všeobecnosti a =< b. Spájanie nebolo optimálne, lebo v jeho súčte sa nachádza
c+a, keby sa však spojila najprv množina A s C tak by bol celkový súčet
väčší, lebo by obsahoval buď c+b alebo c+(a+c).
\end{proof}

\begin{veta}
Pre $\forall n, k: n < k$ platí, že existuje taká postupnosť $n$
vrcholov hyperkocky, medzi ktorými vedie najvyšší možný počet hrán, 
že medzi prvými $k$
vrcholmi tejto postupnosti vedie najvyšší možný počet hrán.
\end{veta}

\begin{proof}
Veta \ref{maximum} a predošlé vety hovoria, ako vyzerajú množiny vrcholov,
v kotrých vedie najvyšší možný počet hrán. Veta \ref{mensia} hovorí o počte
hrán, ktoré vedú medzi dvomi množinami vrcholov rozdelenými podľa nejakej
dimenzie. Veta \ref{izomorfne} zas hovorí, že menšia množina má vo väčšej
izomorfný obraz, v ktorom má každý vrchol menšej dvojice vo väčšej susedný 
vrchol, s ktorým má všetky ostatné dimenzie rovnaké.

Potrebujeme nájsť takú postupnosť prechádzania tejto množiny vrcholov
agentom, že
v každom momente bude medzi dovtedy prejdenou množina vrcholov najvyšší
možný počet hrán pre danú množinu vrcholov.
Teda potrebujeme, aby prejdená množina spĺňala tvrdenie vety \ref{delenie}
 a vety
\ref{mensia}, teda aby sa dala rozdeliť po dimenziách na polovice +/- 1
vrchol a pritom, aby medzi týmito polovicami viedol počet hrán rovný
mohutnosti menšej z nich.

Majme množinu $n$ vrcholov z vety \ref{maximum}, teda vedie v jej
indukovanom grafe najvyšší možný počet hrán. Keď ju delíme podľa vety
\ref{mensia}, tak získame strom delenia. Pri každom vetvení bola od vrchola
v ňom "odrezaná" najviac jedna hrana; ak bol v menšej množine, tak práve
jedna. Ak je jedna množina väčšia, tak v nej existuje vrchol, ktorý nebol
spojený s vrcholom v menšej. My potrebujeme nájsť vrchol, po ktorého
odobratí získame opäť množinu s najvyšším počtom hrán. Súradnice vrchola sú
presne určené jeho polohou v strome delenia podľa dimenzií, keďže vo vete
\ref{mensia} sme určili všetky súradnice, na ktorých záleží. Budeme teda
tento vrchol hľadať prechádzaním tohto stromu. Začneme na najvyššej úrovni a
pôjdeme vždy po niektorej hrane nižšie. Vyberieme si vždy vetvu s väčšou
množinou, ak budú obe množiny rovnaké, tak tú ktorá má v danje dimenzii
súradnicu 0. Týmto sme jednoznačne určili výber vrchola. Tvrdíme, že po jeho
odobratí bude mať množina vrcholov tento strom delenia izomorfný so stromom
delenia množiny na $n-1$ vrcholoch s najvyšším možným počtom hrán v
indukovanom podgrafe. Dve rovnako veľké množiny majú rovnaký podstrom
delenia podľa dimenzií, čo sme už ukázali. Keď sa pozrieme na strom
odvodenia po odobratí vybraného vrchola, zistíme, že nebola porušená veta
\ref{delenie} o rozdiele veľkostí dvoch množín. Keďže sú v každom vetvení
množiny buď rovnaké alebo je menšia tá s vrcholmi so súradnicou 0 v danej
dimenzii, zostáva zachované priradenie súradníc z vety \ref{mensia}.

Našli sme teda spôsob, ako určiť odzadu poradie prechádzania vrcholov
agentom tak, aby prejdená množina vrcholov obsahovala najvyšší možný počet
hrán v indukovanom podgrafe. Pre úplnosť treba dokázať, že agent môže
vrcholy v tomto poradí navštíviť. Teda, že odobratý vrchol susedí s
vrcholom, ktorý bude odobratý najbližšie.
V prípade, že sú obe množiny rovnako veľké bude najprv odobratý prvok z
množiny so súradnicou 0, čím sa táto množina zmenší a následne bude odobratý
prvok z množiny so súradnicou 1 v prvej dimenzii delenia. Keďže obe množiny
majú rovnakú veľkosť výber súradníc vo zvyšných dimenziách dopadne zhodne
pre oba vrcholy. Líšia sa teda práve len v súradnici prvej dimenzie a teda ich
spája hrana. Vyplýva to aj z toho, že menšia množina má izomorfný obraz vo
väčšej.

V prípade, že jedna množina vrcholov je väčšia, tak to nemusí nutne platiť.
Preto urobíme úpravu v postupe výberu vrchola. Ako sme ukázali vo vete
\ref{substitucia} substitúcia súradníc nemá vplyv na počet hrán vedúcich
medzi vrcholmi danej množiny, ak sa vymenia všetky súradnice v danej
dimenzii na opačné. Ak preklopíme súradnice vo všetkých dimenziách, v
ktorých má odoberaný vrchol súradnicu 1, tak nasledujúcim vybraným v takto
substituovaných súradniciach bude jeho sused na najnižšej úrovni stromu.

Na konci môžeme spätne vypočítať skutočné súradnice vrcholov postupnosti.

\iffalse
Ak vynecháme vrchol s nanižším počtom hrán, ktoré doň vedú, získame množinu
s najvyšším počtom hrán medzi $n-1$ vrcholmi?

\fi
\end{proof}

\iffalse
Čiže trade-off medzi počtom značiek a pohybov ... stačí jedna návšteva
vrcholu ... otvorí pri nej všetky porty ... počet značiek = počet 
mimokostrových
(portových) hrán ... 

Čím vyššia dimenzia kocky, tým je menej kostrových hrán oproti portovým ->
stačí menej pohybov na požitie daného množstva značiek.

Ak vrcholov nie je presne toľko, ako v nejakej n-rozmernej kocke tj. ${2}^{n}$
tak ich vyberáme z vrcholov najmenšej kocky s väčším počtom vrcholov tak, že
vezmeme vrcholy nejakej (n-1) rozmernej kocky, ktorá je jej podgrafom, v
ďalšom kroku zvyšok vrcholov vyberieme z ešte nepoužitých tak, aby tvorili kocku
čo najväčšej dimenzie a tak ďalej, kým nevyberieme všetky vrcholy. Každá
n-rozmerná kocka sa dá rozdeliť na dve disjunktné (n-1)-rozmerné kocky.
Každý vrchol takejto kocky je spojený s práve jedným vrcholom z druhej
kocky. Agent teda vždy nájde kocku ktorá je najmenšia taká, že má viac
vrcholov. Jednu z týchto kociek pokryje a zvyšné vrcholy umiestni do druhej
kocky rekurzívnym spôsobom.

Týmto sme vyriešili hľadanie postupnosti k vrcholov s nyjväčším počtom hrán
medzi nimi. Teraz treba zistiť, ako veľká musí byť množina značiek potrebná
na konštrukciu týchto hrán.
Ak má agent minúť čo najmenej značiek pri tvorbe portových hrán, treba aby
ostali porty otvorené čo najkratšie - čas meriame pohybmi agenta.

Počet značiek, ktroé agent potrebuje pre konštrukciu n-rozmernej kocky je
suma dva na i minus jedna pre i od jedna po n. (po konštrukcii jednej kocky
ostávajú otorené porty do druhej a agent potrebuje rovnaký počet značiek ako
na predošlú ... - lenže postupne sa uvoľnujú) pozn. bude to teda asi \#
portov pre štvrtinovú kocku + dva na i munus jedna
\fi 

\section{2-rozmerná kocka}
Jednorozmerná kocka má len jednu hranu, ktorá je zároveň súčasťou každej jej
kostry. Jej konštrukcia je v našom modely triviálna a stojí jednu operáciu -
vytvorenie novej hrany s vrcholom na konci z počiatočného vrchola.

Dvojrozmerná kocka je najmenšia kocka, pri konštrukcii ktorej agent vytvorí
portovú hranu. Je to elementárny prípad. Agentovi stačí jedna portová značka 
a vykonať tri kroky pohyby - operácue NH. Port otvorí v počiatočnom a
poslednom vrchole, čím vznikne posledná potrebná hrana.
\section{3-rozmerná kocka}
\subsection{1 značka}
Pri použití jednej zančky pre porty vie agent trojrozmernú kocku zostrojiť 
na 15 krokov:
OP, NH, NH, NH, OP, OP, S1, NH, NH, OP, OP, S2, S2, NH, OP, OP, S1, S1, NH, OP, OP, S1, S2, S3, OP. Na
menej to nie je možné. Agent potrebuje skonštruovať 5 mimokostrových hrán a na
každú potrebuje aspoň 3 kroky - pohyby medzi otvorením jej prvého a druhého portu.

\subsection{2 značky}
Trojrozmernú kocku pri použití dvoch značiek na porty dokáže agent zostrojiť
na 11 pohybov a menej sa nedá. Operácie agenta:
OP1, NH, NH, OP2, NH, OP1, NH, OP1, NH, OP2, NH, OP2, NH, OP1, OP1, S1, S1, S2, OP1, S1, OP2
Na menej sa nedá, pretože porty treba otvoriť medzi dvojicou vrcholov
líšiacich sa v práve jednej súradnici. Teda medzi otovrením portov jednej
hrany potrebuje agent aspoň dve preklopenia na rovnakej súradnici a jedno na
inej. Vie využiť najviac jedno preklopenie, ktoré nastalo pri konštrukcii
predošlej hrany (pretože na dvojici súradníc ktoré preklápal už existuje pre
vrchol v ktorm sa nachádza dvojrozmerná kocka - práve ju vytvoril), 
čiže na konštrukciu hrany potrebuje najmenej dva ďalšie
pohyby. Na konštrukciu prvej hrany potrebuje tri pohyby. $3 + 2 * 8 = 11$

\subsection{3 značky}
Trojrozmernú kocku agent za použitia troch portových značiek zostrojí na 9
pohybov. Na menej sa nedá.
Operácie agenta: OP1, OP3, NH, NH, OP2, NH, OP1, NH, OP1, NH, OP2, NH, OP2,
NH, OP1, OP3, S3(?), S1, OP2
S použitím menšieho množstva pohybov to agent urobiť nemôže. Kým vznikne
prvá portová hrana - značka sa uvoľní na ďalšie použitie,vytvorí agent tri
nové vrcholy. Teda je vo štvrtom vrchole. Súčet  stupňov predošlých troch je
deväť, z čoho päť zaberú nové hrany. To znamená, že do jedného z týchto
vrcholov sa agent bude musieť vrátiť otvoriť port alebo z neho vytvoriť novú
hranu. Keby tento vrchol bol tretí podľa vzniku, tak sa doň agent môže
vrátiť na jeden ťah. Tu môžu nastať dve možnosti: pôjde tam otvoriť port
alebo vytvoriť novú hranu. Ak otvoriť port, bude mu chýbať značka na štvrtom
vrchole podľa vzniku a bude sa musieť vrátiť aj do neho. Ak otvorí port s
novouvoľnenou značkou na štvrtom vrchole a z tretieho urobí NH. Ocitne sa vo
vrchole v ktorom nemôže otvoriť port, lebo sa nelíši v práve jednej
súradnici od žiadneho z vrcholov v ktorých sú porty pootvárané. Teda v oboch
prípadoch potrebujem ďalšie dva pohyby, ktoré nie sú operáciou NH. Ak sa
agent nevráti hneď do vrchola, v ktorom neotvoril port, bude musieť prejsť
do niektorého z jeho dvoch susedov (lebo z nich už odišiel a z iného vrchola
sa tam nedostane). To sú dva pohyby okrem operácií NH. Počet operácií NH je
konštantný pre daný  graf a rovná sa počtu hrán kostry. V tomto prípade je
to sedem. Teda agent vybavený tromi rôznymi značkami na porty potrebuje
aspoň deväť krokov na konštrukciu trojrozmernej kocky.

\subsection{4 značky}
4 rôzne značky portov agentovi na konštrukciu trojrozmernej kocky stačia:
OP1, OP2, NH, OP3, NH, OP4, NH, OP1, NH, OP1, NH, OP4, NH, OP3, NH, OP2, OP1
Agent vzkoná sedem pohybov, čo je nevyhnutné minimum na konštrukciu
trojrozmernej kocky. Pridanie ďalších značiek už teda nič nevylepší.

\section{4-rozmerná kocka}
\paragraph{1 značka}
Ak má agent skonštruovať portové hrany použitím troch pohybov na hranu a žiadne
ďalšie pohyby, tak musí vo vrchole, kam otvorením portu vznikne nová
portová hrana, otvoriť port znovu, inak by musel vykonať krok navyše.
Portové hrany teda musia tvoriť súvislý ťah. 

Všetkých hrán v štvorrozmernej kocke je 32, pričom kostrových je 15.
Portových hrán je teda 17. Sled portových hrán sa nemôže pretínať, pretože
nemôže existovať vrchol do ktorého nevedie kostrová hrana a každý vrchol má
stupeň 4. Keďže stupeň vrcholov je párny, kostra z hrán vzniknutých
operáciou NH musí byť taká, aby
dva z nich mali nepárny stupeň a zvyšné párny; teda aby tvorila cestu po
všetkých vrcholoch v grafe. Čo je nutná podmienka zostrojenia
jediného ťahu zo všetkých zvyšných hrán - portových hrán. 
Kostra má aposň dva listy a stupeň listov je nepárny. 
Ak teda kostra nemá mať väčší počet listov, musia mať
zvyšné vrcholy stupeň dva. Kostra je teda tiež cesta. Ukážeme, že je možné
zostrojiť štvorrozmernú kocku na $3 * 17 = 51$ pohybov s jedinou značkou pre
porty. 
\begin{veta}
Existuje konštrukcia štvorrozmernej kocky, pri ktorej agent vykoná 51 krokov
a použije tri rôzne značky pre porty.
\end{veta}
\begin{proof}
Aby sme skrátili zápis, označíme dimenzie - rozmery prirodzenými číslami od 1 po 4 a
uvedieme len v ktorej dimenzii agent preklopil súradnicu svojej polohy pri
pohybe. Pre lepšiu prehľadnosť zoskupíme zmeny dimenzií týkajúce sa konštrukcie
rovnakej hrany do zátvoriek po troch: (121), (343), (121), (242), (434),
(323), (212), (131), (141), (414), (313), (323), (212), (232), (141), (131),
(343).

Čiže postupnosť operácií by vyzerala takto: OP, 3 $\times$ NH, OP, OP, 3 $\times$
NH, OP, OP, 3 $\times$ NH, OP, OP, S2, S2, S2, OP, OP, S2, NH, NH, OP, OP,
...
\end{proof}

\section{Konštrukcie kociek vyšších rozmerov}


\iffalse

\fi