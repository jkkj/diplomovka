

%\cleardoublepage
\phantomsection
\addcontentsline{toc}{chapter}{Všeobecná konštrukcia}
\chapter{Všeobecná konštrukcia}\label{chap:intro}

\section{Definícia}
\begin{defin}
Model sa skladá z grafu a agenta. Graf má nemeniteľné lokálne číslovanie koncov hrán vo
vrchole, podľa poradia vzniku. Koniec hrany, ktorý vznikol ako prvý, má
číslo 1. Prvý koniec vznikne spolu s vrcholom, zvyšné otvorením portu.
Počiatočný graf obsahuje práve jeden izolovaný vrchol a nič viac. Akonáhle
sú otvorené dva porty s rovnakými značkami, vznikne medzi nimi nová hrana a
tieto porty tým zaniknú.
\\
Agent môže vo vrchole, v ktorom sa nachádza, vykonávať tieto operácie: pohnúť sa po hrane do susedného
vrcholu, otvoriť port so značkou, vytvoriť hranu do nového vrchola v ktorom sa následne
ocitne. Agent rozlišuje hrany podľa lokálneho číslovania ich koncov vo
vrchole, kde sa práve nachádza. Agent vie ktorý koniec patrí hrane po ktorej
prišiel do vrcholu. Agent začína vo vrchole počiatočného grafu.
\\
Agent pri vykonávaní algoritmu konštrukcie grafu nemá obmedzenú výpočtovú
silu ani pamäť. Množina značiek portov je konečná a vopred daná. Efektivita
algoritmu konštrukcie daného grafu sa pri tomto modely meria počtom pohybov
agenta (tj. sumou počtu návštev cez všetky vrcholy) pri danej množine značiek
portov, tento počet sa snažíme minimalizovať.
\end{defin}

\begin{ozn}
Označenie operácií agenta:
- nová hrana = NH
- otvoriť port (s číslom n) = OP(n)
- prejsť sa po hrane s lokálnym číslom konca k = Sk
\end{ozn}

\begin{ozn}
Agent začína v grafe, ktorý sme už nazvali počiatočný graf - v našom prípade
obsahuje jeden izolovaný vrchol. Graf, ktorý má agent skonštruovať, budeme
volať cieľový, konečný alebo konštrupvaný. Graf existujúci v danom kroku
konštrukcie v modely budeme volať aktuálny graf alebo len graf.
\end{ozn}

\begin{pozn}
Pri práci s neorientovanými grafmi používame pojem "hrana vedie z vrchola
$v$
do vrchola $u$" len na rozlíšenie týchto dvoch vrcholov od seba.
\end{pozn}

\begin{ozn}
Nad prvkami grafu, konštruovaného jedným agentom,
  sú to najmä vrcholy, hran i porty, existuje prirodzené úplné
usporiadanie podľa poradia ich vzniku, podľa ktorého môžeme o jednom prvku
vyhlásiť, či je starší, mladší alebo rovnako starý ako iný prvok. Platí, že
hrana je rovnako stará ako jej druhý port. Všeobecne prvky vzniknuté jednou
operáciou sú rovnako staré.
\end{ozn}

\begin{ozn}
Krokom agenta budeme nazývať vykonanie operácie, pri ktorej sa zmení vrchol,
v ktorom sa agent nachádza.
\end{ozn}

\begin{ozn}
Portová hrana je hrana, ktorá vznikla operáciou OPk, kde k je značka portu,
ktorého otvorením hrana vznikla.
\end{ozn}
\begin{ozn}
Kostrová hrana je hrana, ktorá vznikla operáciou NH.
\end{ozn}
\begin{pozn}
Ako neskôr ukážeme - "kostrové hrany" tvoria kostru grafu.
\end{pozn}

\begin{lem}
Agent na ceste do vrchola v grafe musí ísť po hranách, ktoré po jeho
návšteve v grafe existujú.
\end{lem}

\begin{proof}
Žiadna operácia agentovi nedovoľuje priamo prejsť medzi dvoma vrcholmi, ktoré
nespája hrana. Môže však vytvoriť kostrové hrany a potom otvorením portu
vznikne hrana, ktorá mu umožní dostať sa do vrcholov grafe, ktoré vytvoril
predtým.
\end{proof}

\begin{lem}
Ak agent odíde z vrcholu v ktorom je súčet počtov incidentných hrán a
otvorených portov nižší, ako je počet hrán s ktorými má byť vrchol
incidentný v cieľovom grafe, tak sa agent doň musí neskôr vrátiť.
\end{lem}

\begin{proof}
Neexistuje postupnosť operácií agenta, pomocou ktorej môže agent vytvoriť
hranu do
existujúceho vrchola bez jeho návštevy, ak v ňom nie je otvorený port pre
túto hranu. Operácia NH vytvorí z vrchola v ktorom sa nachádza do nového
vrcholu, ktorý predtým v grafe neexistoval a operáciou OPk môže agent
vytvoriť hranu z vrchola v ktorom sa nachádza do vrchola v ktorom je
otvorený port so značkou k. Operácipu OPk nové hrany nevznikajú. Teda ak sa
v danom vrchole agent nenachádza a nemá v ňom otvorený port pre hranu,
ktorú ide skonštruovať, nemôže to urobiť bez návštevy tohto vrchola.
\end{proof}

\begin{lem}
Hrany vytvorené operáciou "nová hrana" tvoria kostru grafu.
\end{lem}
\begin{proof}
Okrem inicializačného vrcholu, všetky vrcholy vzniknú operáciou "nová hrana"
(NH). Každý vrchol okrem inicializačného je spojený hranou s lokálnym číslom
1 so starším vrcholom (vrchol, ktorý vznikol skôr). Po týchto hranách sa dá
z každého vrchola dostať do inicializačného, teda podgraf tvorený hranami
pochádzajúcimi z oprácií NH je súvislý. Ich počet je zároveň o jeden menší ako počet
vrcholov, teda ide o kostru.
\end{proof}
\begin{lem}
\label{uzavrety_sled}
Hrany, ktoré agent prejde od otvorenia prvého portu hrany po otvorenie druhého,
tvoria v grafe s novovzniknutou hranou uzavretý sled.
\end{lem}
\begin{proof}
Je to dané operáciami, ktoré môže agent vykonať. Agent totiž prechádza len
po hranách v grafe. Z vrchola ide do incidentnej hrany a z hrany do
incidentného vrchola. Pričom žiadna hrana nemôže zaniknúť, teda medzi otvorením
dvoch portov prejde agent po nejakom slede v grafe. Novovzniknutá hrana
spája koncový a začiatočný bod tohot sledu; tvorí s ním teda uzavretý sled.
\end{proof}
\begin{pozn}
Zo sledu z vety \ref{uzavrety_sled} vieme vybrať kružnicu.
\end{pozn}
\begin{lem}
\label{minimum}
Minimálny počet hrán, ktoré agent potrebuje prejsť pri konštrukcii hrany pomocou
portov je rovný alebo väčší ako dĺžka najmenšej kružnice, na ktorej sa táto
hrana nachádza.
\end{lem}
\begin{proof}
Nájdeme menší takýto sled ako je najkratšia kružnica. Tento tvorí s
novovzniknutou hranou uzavretý sled. Z neho vieme vybrať
kružnicu, čo je spor s minimálnosťou najkratšej kružnice.
\end{proof}
\begin{lem}
Ak má agent k dispozícii dostatočný počet značiek portov, 
tak sa problém efektívnej konštrukcie grafu
redukuje na hľadanie najkratšieho sledu v ktorom sú všetky vrcholy. Pri
hamiltonovskom grafe ide o hamiltonovskú kružnicu.
\end{lem}
\begin{proof}
Agent skonštruuje graf takto: prechádza vrcholmi, podľa najkratšieho sledu
obsahujúceho všetky vrcholy, pričom mu sačí jedna návšteva každého vrcholu,
 v ktorom
otvorí potrebný počet portov príslušných značiek. Ak má v slede prejsť do
vrcholu, ktorý ešte neexistuje, použije operáciu NH.
\end{proof}
\begin{lem}
Vrchol v grafe musí agent navštíviť minimálne $\frac{deg(v)}{|Z| * 2} $ krát,
 kde Z je množina značiek pre porty, ktorú ma agent k dispozícii. Pričom do
stupňa vrchola nerátame pre účely tejto vety slučky.
\end{lem}
\begin{proof}
Pri každej návšteve môže agent otovrením portov so všetkými značkami 
vytvoriť nové hrany do vrcholov, v ktorých sú otovrené príslušné porty 
a následne otvoriť toľko
portov s navzájom rozdielnymi značkami, koľko má k dispozícii značiek
(dvojica portov rovnakej značky v jednom vrchole by vytvorila slučku).
\end{proof}
\begin{pozn}
Ako ukážeme neskôr, pri obmedzenom množstve značiek portov môže byť
výhodnejšie neminúť všetky značky pre porty v jednom vrchole, aj za cenu
opakovanej návštevy. Výhodnejšie v tomto prípade znamená, že agent pri danej
konštrukcii vykoná menej krokov.
\end{pozn}


\subsection{Jedna značka pre porty}

\begin{lem}
Najmenej krokov agent na konštrukciu konkrétnej hrany medzi už existujúcimi
vrcholmi použije, ak pôjde do najbližšieho z nich, otvorí v ňom port a
následne najkratšou cestou po už existujúcich hranách prejde do druhého
vrcholu a otvorí port v ňom.
\end{lem}
\begin{proof}
Agent má k dispozícii len jednu značku pre porty. Teda počas konštrukcie
hrany medzi už existujúcimi vrcholmi, žiadne nové hrany v grafe,
 ktoré by ležali na ceste medzi koncami tejto hrany nevznikn.
 Jediné nové hrany môžu vzniknúť operáciou NH, ale
tieto by viedli do "slepej uličky" z ktorej by sa musel agent vracať a je
zrejmé, že ak ich konštrukciu vynechá, z pohľadu najmenšieho počtu ťahov na
konštrukciu tejto portovej hrany nič nepokazí. V neorientovanom grafe
pri jednej značke pre port je najkratšia cesta od otvorenia prvého po
otvorenie druhého portu hrany rovnaká bez ohľadu na to, 
v ktorom vrchole agent otvorí prvý port hrany.
Jediné, čo sa mení, je počet krokov po otvorenie prvého z portov.
Aby bol tento a teda aj celkový počet krokov čo najkratší, pôjde agent do
najbližšieho vrchola.
\end{proof}

\begin{lem}
Agent, ktorý má k dispozícii len jednu značku pre porty, 
musí prejsť pri konštrukcii grafu aspoň k krokov, kde 
$k =min_{i=1}^{y}\{\sum_{e \in G(E); e \notin G(K_{i})} j(e)-1  \}$, 
kde K je kostra grafu G; y je počet navzájom rôznych kostier garfu G; j(e) je najmenšia kružnica v
grafe obsahujúca hranu e a $K_{i}$ je i-ta kostra, pričom $K_{i} \equiv K_{j}
\iff i \equiv j$
\end{lem}
\begin{proof}
Podľa vety \ref{minimum} musí agent prejsť pri konštrukcii hrany aspoň dĺžku
minimálnej kružnice bez jednej - bez hrany ktorú tým konštruuje. Keďže má k
dispozícii len jednu značku pre porty, ďalšiu hranu môže začať konštruovať
až keď dokončil predošlú. Pri každej hrane prejde teda aspoň dĺžku
najmenšej kružnice, v ktorej sa táto hrana nachádza mínus jedna. Dokopy
prejde agent ich súčet.
\end{proof}



\begin{lem}
Po použití operácie NH, ktorou vznikne hrana h, sa agent už nedostane do
už existujúceho vrchola bez následného použitia
operácie OP, ktorá vytvorí hranu do vrcholov starších ako h,
alebo bez opätovného prejdenia hrany h.
\end{lem}
\begin{proof}
Vrchol v, ktorý vznikne operáciou NH je so zvyškom grafu spojený len hranou
h, ktorá tvorí most do vzniku hrany operáciou OP medzi komponentmi na oboch
stranách tohto mostu. Teda jediný spôsob ako prejsť medzi týmito dvomi
časťami grafu je po moste, ktorý ich spája - po hrane h.
\end{proof}



\subsection{Dve a viac značiek pre porty}
Ušetriť oproti situácii, keď agent môže použiť len jednu značku pre porty
možno vtedy, ak konštruovaný graf obsahuje cykly, kotré majú spoločné hrany.
Vtedy, ak agent pri konštrukcii niektorej hrany, 
prechádza vrcholom s ktorým bude incidentná niektorá
portová
hrana, ktorú bude konštruovať po tejto,
 v ňom otvorí port. Následne sa do tohto vrchola už nemusí pri konštrukcii
tejto nasledujúcej portovej hrany vracať otvoriť port, ak existuje
kratšia cesta do druhého koncového vrchola tejto hrany. V metrike
vzdialeností, kde sa počíta počet hrán najkratšej cesty medzi dvomi vrcholmi
ako ich vzdialenosť platí trojuholníková nerovnosť. Z tohto dôvodu agent
nevykoná viac krokov pri najlacnejšej (na najmenej krokov) 
konštrukcii portovej hrany ak jeden port tejto hrany otvorí skôr oproti
konštrukcii kde ho otvorí neskôr.
Ako neskôr ukážeme, nie vždy spôsobí snaha skonštruovať jednu hranu na čo
nejmenej krokov zlepšenie celkového počtu krokov agenta počas konštrukcie
cieľového grafu.


\section{Mriežka}
\begin{defin}
Mriežka veľkosti $ m \times n$ je graf $P_{m} \times P_{n}$
\end{defin}
\begin{defin}
Cyklická mriežka veľkosti $m \times n$ je graf $C_{m} \times C_{n}$.
\end{defin}

Ďalej sa budeme zaoberať mriežkou a cyklickou mriežkou. Na to, aby sme sa v
týchto dvoch grafoch dokázali dobre orientovať, zavedieme pojem dimenzie a
jej súradnice. Pričom využujeme, že oba grafy sú produktmi kartézskeho
súčinu.
\begin{ozn}
Pojem dimenzie v produkte kartézskeho súčinu $G$, grafov $U$ a $V$ 
zavedieme prirodzene vzhľadom
na označenie vrcholov produktu ako usporiadanej dvojice $(u,v)$, kde $u \in
U$ a $v \in V$. i-tou dimenziou vrchola produktu tohto súčinu bude i-ty člen
tejto dvojice. Súradnicou v dimenzii i je prvok na i-tej pozícii v
usporiadanej dvojici. Pre zjednodušenie označíme vrcholy pôvodných grafov
prirodzenými číslami od 1 po m respektíve od 1 po n. Vrcholy produktu
kartézskeho súčinu týchto dvoch grafov budú pre dosť veľké m,n napríklad
$(2,4)$, $(5,17)$, $(9,1)$, atď.
\end{ozn}

\begin{lem}
Pri konštrukcii portovej hrany v mriežke aj v cyklickej mriežke urobí agent medzi otvorením prvého a druhého
portu najmenej tri kroky.
\end{lem}
\begin{proof}
Najmenší cyklus v oboch grafoch má dĺžku štyri. Teda medzi otvorením dvoch portov
hrany vykoná agent najmenej tri pohyby po hranách. Buď prejde po už
existujúcich alebo vytvorí nové operáciou NH.
\end{proof}

\subsection{Jedna značka pre porty}

\begin{lem}
Pri ľubovoľnej konštrukcii cyklickej mriežky $n \times m$ s jednou značkou
pre porty, existuje aspoň jedna hrana pri konštrukcii ktorej agent vykoná 
aspoň $n$ krokov
a aspoň jedna portová hrana pri konštrukcii ktorej agent vykoná aspoň $m$
krokov.
\end{lem}
\begin{proof}
Pozrime sa na vrcholy, ktoré majú rovnakú súradnicu v druhej dimenzii,
označme ju $k$ a množinu týchto vrcholov označme $V_{k}$. 
Ukážeme, že pre každé $k$ má jeden vrchol z množiny $V_{k}$ v
podgrafe indukovanom vrcholmi $V_{k}$ vzdialenosť od jedného svojho suseda $n$.
Pozrime sa na postupné vytváranie vrcholov $V_{k}$.
\end{proof}


\begin{lem}
Mriežku $m \times n$ je možné s použitím jedinej značky pre port zostrojiť
pomocou $3mn + m$ pohybov. Kde m je menší z rozmerov.
\end{lem}
\begin{proof}
Agent vykonáva: OP,NH,NH,NH,OP,(n-2)krát (OP,S1,NH,NH,OP),S1,NH,HN,OP čím
vytvorí prvý riadok mriežky. Následne vykoná
(m-1)krát (NH,OP,S1,S2,NH,OP,(n-1)krát (OP,S1,S2,NH,OP))
\end{proof}
\begin{lem}
TODO ... na menej to nejde ...
\end{lem}
\begin{proof}
TODO
\end{proof}




\section{N-rozmerná kocka}
Vrcholy spojené hranou sa líšia práve v jednej súradnici. Každej hrane z
vrcholu možno priradiť súradnicu v ktorej sa jej druhý vrchol od tohto
odlišuje. Toto priradenie je injekcia.
Susedia vrchola sa navzájom líšia v práve dvoch súradniciach. Pri pohybe po
hrane mením jednu súradnicu svojej pozície. Z tohto vyplývajú obmedzenia
pre úsporu pohybov pridávaním značiek pre porty. Napríklad pri použití dvoch
značiek pre porty potrebuje agent na vytvorenie dvoch nových portových hrán
aspoň 5 ťahov. Úspora nastáva, keď port pre druhú hranu otvorím už počas
cesty za otvorením portu prvej hrany.

\begin{lem}
Dva vrcholy, ktoré sa navzájom líšia v práve dvoch súradniciach majú práve dvoch
spoločných susedov.
\end{lem}
\begin{proof}
Susedné vrcholy sa líšia práve v jednej súradnici. Spločný sused dvoch
vrcholov sa líši v práve jednej súradnici od každého z nich. Teda má všeky
súradnice rovnaké ako jeden z nich a líši sa v jedej z tých dvoch. Teda ju
má rovnakú ako druhý vrchol. Druhú z týchto súradníc má potom odlišnú od
druhého vrchola. Čiže sa od každého z nich líši v práve jednej súradnici.
Toto dáva práve dva rôzne vrcholy, ktoré susedia z oboma. Vrchol, ktorý
sa od jedného z vrcholov líšiacich sa v dvoch súradniciach líši v nejakej
inej, sa od druhého líši v troch súradniciah a teda s ním nesusedí.
\end{proof}

Otázkou ostáva, koľko hrán vie agent vytvoriť počas jednej návštevy vrcholu
otváraním portov? 
\begin{lem}
Pri jednej návšteve vrcholu môže agent vytvoriť najviac $(frac{n-1}{2}) > \#dimenzii$
hrán otvorením portu.
\end{lem}
\begin{pozn}
Rátajú sa hrany, ktoré vzniknú hneď tj. pri ktorých agent otvorením portov
vytvorí druhý (posledný) koniec.
\end{pozn}
\begin{proof}
Je zrejmé: že vo všetkých vrcholoch, kam majú tieto hrany
viesť, už agent bol; medzi dvomi vrcholmi vedie najviac jedna hrana. Z tohto
a predošlej vety vyplýva, že po návšteve n vrcholov môže agent otvorením
portu vytvoriť z vrcholu, kde sa nachádza, najviac $(frac{n-1}{2}) > \#dimenzii$ hrán. Z
týchto $(frac{n-1}{2})$ vrcholov sa od neho každý líši v inej súradnici,
lebo ak by sa dva líšili v rovnakej, museli by to byť totožné vrcholy. 
\end{proof}
\begin{lem}
Existuje taká postupnosť $n$ vrcholov, že pri návšteve posledného agent
vytvorí $(frac{n-1}{2}) > >\#dimenzii$ hrán.
\end{lem}
\begin{proof}
Ak vezmeme takúto postupnosť odzadu, vyzerala by až na substitúciu dimenzií
takto: 1, 2, 1, 3, 2, 4, 3, 5, 4, 6, 5, ...
\end{proof}
\begin{pozn}
Avšak ľahko vidieť, že žiadne iné portové hrany by medzi vrcholmi v tejto
postupnosti vzniknúť nemohli.
\end{pozn}
Skúsme teda nájsť postupnosť (cestu v hyperkocke) s najväčším možný počtom 
portových hrán medzi jej vrcholmi.
\begin{ozn}
O hrane povieme, že vedie v dimenzii k, ak sa vrcholy, ktoré spája, líšia v
práve tejto dimenzii. 
\end{ozn}
Je jasné, že v akomkoľvek súvislom slede vrcholov povedie hrana
medzi dvojicami nasledujúcich vrcholov, ide o hrany po ktorých agent prejde
z vrcholu do vrcholu.
Portová hrana môže vzniknúť medzi vrcholmi, ktoré sa líšia v práve jednej
súradnici, lebo iné hrany v hyperkocke nie sú. 
\begin{ozn}
Poloha agenta sa skladá z jeho polôh v jednotlivých dimenziách kocky. v
každej dimenzii môže nadobúdať jednu z dvoch hodnôt. Označme ich $0$ a $1$.
Pri prechode po hrane agent preklopý jednu z týchto hodnôt v práve jednej
dimenzii na druhú.
\end{ozn}

\begin{lem}
V ľubovoľnej postupnosti hrán medzi vrcholmi hyperkocky spojenými hranou, sa
nachádza párny počet hrán v každej dimenzii okrem dimenzie v ktorej sa tieto
vrcholy líšia a teda v nej vedie hrana medzi nimi.
\end{lem}
\begin{proof}
Takéto dva vrcholy sa líšia v práve jednej dimenzii. Ak agent zmenil pri 
pohybe svoju polohu v nejakej dimenzii nepárny počet 
krát, tak je v tejto dimenzii v inej polohe ako bol na začiatku. Ak to nie je 
dimenzia v ktorej sa líši koncový vrchol od začiatočného - agent sa
nenachádza v koncovom vrchole, lebo polohy agenta a tohto vrcholu nie sú
rovnaké.
\end{proof}

\begin{ozn}
Ak viem súradnice vrchola, môžem postupnosť ťahov agenta zapísať ako postupnosť 
zmien jeho súradníc v jednotlivých dimenziách. 
\end{ozn}
Ak sa nachádza v takejto postupnosti medzi
vrcholmi, medzi ktorými vzniká portová hrana dvakrát hneď za sebou číslo
nejakej dimenzie, znamená to, že agent šiel tam a späť a tieto dva prvky
môžeme z postupnosti vylúčiť, lebo je to z pohľadu tejto dvojice vrcholov
zbytočný pohyb. Ak takto zredukujeme niektorú postupnosť, získame inú, v
ktorej ide agent po podmnožine hrán z pôvodnej a na rovný alebo menší počet
pohybov.
\begin{pozn}
Pripomeniem, že hľadáme taký sled, že počet portových hrán medzi jeho
vrcholmi je na daný počet značiek pre porty a počet portových hrán, ktoré
vzniknú medzi jeho vrcholmi, najkratší možný.
\end{pozn}

Ak má agent dostatok značiek pre porty, nezaujíma nás poradie v akom vrcholy
postupnosti navštívil. Úlohu teda môžeme zredukovať na hľadanie takej množiny
vrcholov veľkosti m, ktorá tvorí zo všetkých takto veľkých množín najmenší rez
od zvyšku grafu. Pri konštantnom stupni vrcholov to znamená, že v takej vedie 
najviac hrán medzi vrcholmi v tejto množine.
Keď sa pozrieme na vrcholy a pravidlá ich spájania hranami, zistíme, že
tieto vedú len v nejakej kocke a čím viac rozmerov kocka má, tým viac hrán
vedie medzi jej vrcholmi. Teda aj v n-rozmernej kocke povedie najviac hrán
medzi vrcholmi takej množiny, ktorej indukovaný graf je izomorfný kocke.
Vrchol k-rozmernej kocky sa nachádza v k (k-1)-rozmerných kockách, ktoré sú
jej podgrafmi. Teda budeme hľadať množinu m bodov, ktoré sa dajú
pokryť kockami čím vyššieho rozmeru a nemusíme si všímať kocky ktoré sú
podgrafmi týchto kociek. tieto kocky sa však môžu prekrývať.

\begin{lem}
S každou hyperkockou v ktorej vrchol neleží ho spája najviac jedna hrana.
Ktorá vedie v inej dimenzii ako sú dimenzie tejto hyperkocky.
\end{lem}
\begin{proof}
Každý vrchol hyperkocky má hrany vo všetkých jej dimenziách. Vo zvyšných
dimenziách sú vrcholy hyperkocky totožné, lebo sa dva susedné líšia práve v
jednej. Vrchol mimo hyperkocky spojený s vrcholom v nej sa od neho líši v
inej dimenzii a v tejto sa líši teda aj od všetkých ostatných. Zároveň je vo
všetkých dimenziách hyperkocky totožný s vrcholom z nej a teda sa lííši v
aspoň jednej dimenzii hyperkocky od jej ostatných vrcholov a má teda s nimi
aspoň dve rodielne dimenzie. Nemôže ho preto s nimi spájať hrana.
\end{proof}

\begin{lem}
Bijektívna substitúcia dimenzií na všetkých vrcholoch podgrafu hyperkocky zachováva
hrany medzi vrcholmi. Rovnako aj preklopenie súradníc v niektorej dimenzii.
\end{lem}
\begin{proof}
Ak sa dva vrcholy líšia práve v jednej dimenzii, budú sa v práve jednej
dimenzii líšiť aj po tejto substitúcii, lebo dimenzie v ktorých boli rovnaké
sa zobrazia na rovnaké a tá v ktorej sa kíšili sa u oboch zobrazí na rovnakú
a budú sa v nej líšiť aj naďalej; teda medzi nimi povedie hrana.
Pri preklopení súradníc sa medzi vrcholmi zachovajú rovnosti a rozdiely v
danej dimenzii, teda aj hrany.
\end{proof}

\begin{lem}
Vrcholy súvislého podgrafu hyperkocky s počtom vrcholov m sa navzájom líšia
v najviac m-1 súradniciach.
\end{lem}

\begin{proof}
Vezmime kostru tohto podgrafu. Táto má m-1 hrán. Každá vedie v niektorej
dimenzii. Cesta v tejto kostre udáva v ktorých dimenziách sa daná dvojica
jej koncových vrcholov líši. Všetky cesty v tejto kostre idú cez najviac m-1
rôznych dimenzií, čo je jej počet hrán. V ostatných dimenziách sa dané
vrcholy nelíšia, majú v nich teda navzájom rovnaké súradnice.
\end{proof}

\begin{lem}
Podmnožina m bodov hyperkocky, ktoré indukujú graf s najvyšším počtom hrán je
súvislá.
\end{lem}

\begin{proof}
Nech existujú dva komponenty. Na jednom z nich urobíme 
postupné popreklápanie súradníc tak, aby susedil niektorý jeho bod s bodom
v druhom komponente a teda medzi nimi viedla hrana. Čiže vyberieme jednu
dimenziu v ktorej sa líšia a vo všetkých ostatných dimenziách, kde majp
rozdielne súradnice preklápame postupne súradnice jedného z vrcholov. 
V prípade, že počas tohot postupu vznikne hrana spájajúca vyhrané dva
komponenty, skončíme skôr. Pôvodné hrany sa
zachovajú a jedna nová pribudne, čo je spor s najvyšším počtom hrán.
Tieto kroky môžeme urobiť bez toho, aby sa dva vrcholy zobrazili na seba.
Ak budeme robiť preklápanie súradníc po jednotlivých dimenziách,
v tom prípade, ak by sa nejaký vrchol transformovaného komponentu
podgrafu zobrazil na niektorý vrchol druhého komponentu, musela by 
medzi nimi viesť predtým hrana, teda stav v ktorom by boli komponenty
prepojené a preklápanie by teda už nepokračovalo.
\end{proof}

Potrebujeme vybrať m vrcholový indukovaný podgraf hyperkocky tak, aby počet 
hrán v ňom bol maximálny možný.

\begin{lem}
Vrcholy podgrafu hyperkocky sa dajú rozdeliť na dve množiny A,B podľa súradnice vo
vybranej dimenzii. Počet hrán vedúcich z vrcholov v jednej množine do
vrcholov v druhej množine je rovný alebo menší ako mohutnosť menšej z nich.
Čiže $|\{e|e = (a,b);a \in A; b \in B\}| <= min\{|A|,|B|\}$
\end{lem}
\begin{proof}
Každý vrchol má v jednej dimenzii najviac jednu hranu a hrana vedie v
danej dimenzii medzi dvomi vrcholmi vtedy ak je to jediná dimenzia v ktorej
majú rozdielne súradnice. V inej dimenzii medzi A a B hrana viesť nemôže,
lebo by musela viesť medzi vrcholmi líšiacimi sa v najmenej dvoch 
dimenziách - v dimenzii hrany a v dimenzii podľa súradníc ktorej 
sú A a B vytvorené. Teda hrany medzi A a B môžu viesť len v jednej dimenzii
a do menšej z A,B môže viesť najviac toľko hrán v jednej dimenzii, koľko má
vrcholov.
\end{proof}

\begin{lem}
\label{izomorfne}
Keď vezmeme z podgrafu hyperkocky len vrcholy, ktoré majú hranu v jednej
vybranej dimenzii a označíme nimi indukovaný podgraf G. Vieme vrcholy G
rozdeliť na dva izomorfné indukované podgrafy hyperkocky $G_{1}$ a $G_{2}$
podľa súradnice vrchola v tejto dimenzii.
\end{lem}

\begin{proof}
Vrcholy, medzi ktoými vedie hrana sa líšia v práve jednej dimenzii, v
ostatných dimenziách sú rovnaké. Keď teda vezmeme hrany vedúce medzi 
$G_{1}$ a $G_{2}$ a vytvoríme podľa nich bijektívne zobrazenie medzi týmito
podgrafmi tak, aby sa dva vrcholy medzi ktorými vedie hrana vo vybranej
dimenzii zobrazili na seba, dostaneme izomorfizmus. Ak totiž viedla medzi
dvomi vrcholmi hrana, bude hrana viesť aj medzi ich obrazmi, lebo tieto majú
rovnaké súradnice ako ich vzory vo všetkých dimenziách okrem jednej, kde sa
oba obrazy od svojich vzorov líšia a teda majú súradnicu v tejto dimenzii 
rovnakú; obaja sú v komponente, kde majú túto súradnicu rovnakú všetky
vrcholy. Ak sa teda líšili v jednej súradnici vzory, budú sa aj obrazy a ak
sa líšili vzory vo viacerých súradniciach, budú sa v rovnakých súradniciach
líšiť aj obrazy. Zobrazenie  je teda izomorfizmus.
\end{proof}

\begin{lem}
\label{delenie}
Existuje indukovaný podgraf $G$ hyperkocky na m vrcholoch, ktorý je možné postupne rozdeliť
podľa dimenzií až po jednovrcholové indukované podgrafy tak, že v každom
kroku rozdelíme vrcholy každého podgrafu s počtom vrcholov aspoň dva do
dvoch podmnožín $A$ a $B$. Takých, že $|A| - |B| <= 1$.
\end{lem}

\begin{proof}
Dôkaz bude konštrukčný. Budeme postupne deliť m vrcholovú množinu i ďalšie z
nej vzniknuté množiny na menšie, až kým nedostaneme m jednoprvkových množín
vrcholov. Pritom budeme postupne určovať súradnice jednotlivých vrcholov.

Budeme postupovať postupne po dimenziách, až kým budúd všetky množiny
jednoprvkové. Vezmeme množinu $M$, rozdelíme ju na dve časti $M_{1}, M_{0}$
také že $|M_{1}| - |M_{0}| <= 1$. Vrcholy v $M_{1}$ budú mať v tejto
dimenzii súradnicu 1 a vrcholy v $M_{0}$ budú mať v tejto dimenzii súradnicu
0. Ak je množina jednoprvková a delenie niektorých iných ešte pokračuje,
dostane v každej ďalšej dimenzii súradnicu 1. Ak má kocka viac dimenzií ako
je potrebných, aby obsiahla všetkých m vrcholov podgrafu, tak môžu dostať
tieto vrcholy v týchto zvyšných dimenziách ľubovoľné súradnice, ale pre
všetky vrcholy musí byť súradnica v každej z týchto dimenzií rovnaká.
Tým sme zistili súradnice vrcholov vo všetkých dimenziách, podľa ktorých
podgraf delíme. Pre účely tejto vety na súradniciach ostatných dimenzií
nezáleží a môžu byť ľubovoľné.
\end{proof}

\begin{lem}
Pri delení z vety \ref{delenie} vedie medzi množinami vrcholov $M_{1}$ a $M_{0}$,
ktoré vznikli rozdelením jednej množiny M podľa tohto delenia, práve toľko
hrán, ako je mohutnosť menšej z týchto množín. Pričom, ak má kocka viac dimenzií ako
je potrebných, aby obsiahla všetkých m vrcholov podgrafu, tak môžu dostať
tieto vrcholy v týchto zvyšných dimenziách ľubovoľné súradnice, ale pre
všetky vrcholy musí byť, na rozdiel od predošlej vety, súradnica
 v každej z týchto dimenzií rovnaká.
\end{lem}

\begin{proof}
Stačí ukázať, že menší z indukovaných podgrafov má vo väčšom izomorfný obraz
podľa vety \ref{izomorfne}. Vtedy vedie z každého vrchola v menšom podgrafe
hrana do nejakého vrchola vo väčšom podgrafe a týchto hrán je presne toľko, koľko je
vrcholov menšieho podgrafu.
Teda potrebujeme dokázať, že vo väčšom podgrafe existuje skupina vrcholov,
ktorej budú súradnice v dimenziách pridelené rovnako ako vrcholom v menšom
podgrafe. Prideľovanie súradníc je deterministické, teda rovnako veľkým
množinám vrcholov v rovnakej fáze budú poprideľované súradnice vo zvyšných
dimenziách rovnako. Na menší a väčší podgraf sa rozdelia grafy s nepárnym
počtom vrcholov.
Použijeme úplnú indukciu. Báza indukcie bude množina veľkosti tri. Menšie
podgrafy sú už len veľkosti dva a jedna. Jednovrcholový sa už ďalej nedelí a
dvojvrcholový sa rozdelí na dva jednovrcholové.
Predpokladáme, že pre delenie množín s mohutnosťou menšou ako n platí veta.
Dokážeme, že potom musí platiť aj pre množinu M mohutnosti n: Ak je n párne,
dá sa množina rozdeliť na dve množiny rovnakej mohutnosti, ktorých vrcholom
sa zvyšné dimenzie poprideľujú rovnako. Teda každý vrchol v jednej bude mať
iný vrchol v druhej množine s ktorým sa bude líšiť v práve jednej dimenzii
a to v tej podľa ktorej sme M rozdelili.
Ak n je nepárne, vzniknú dve množiny $M_{1}$, $M_{2}$. Obe z nich už
podľa IP spĺňajú tvrdenie vety. Rozdelia sa každá na dve množiny, pričom z
množiny s párnou mohutnosťou vzniknú dve rovnako veľké množiny
$M_{11}$,$M_{12}$ a množina $M_{21}$
tejto veľkosti vznikne aj z množiny s nepárnou mohutnosťou. Druhá množina
$M_{22}$ z tejto množiny bude mať o prvok viac alebo menej.
Prvé tri množiny sa už budú deliť rovnako, teda viem povyberať trojice
vrcholov, každý z inej množiny, ktoré budú mať vo zvyšných dimenziách
rovnaké súradnice. Medzi dvomi vrcholmi z $M_{11}$,$M_{12}$, označme ich 
$m_{11}$,$m_{12}$,  povedie hrana,
lebo sa líšia práve v jednej dimenzii. Tretí vrchol, označme ho $m_{21}$
 bude spojený hranou s
niektorý z tých dvoch, pretože pri druhom delení jeden z nich dostal rovnakú
súradnicu ako on v dimenzii druhého delenia a teda sa líšia iba v dimenzii
prvého delenia množiny M. Ak je prvok $m_{21}$ spojený s prvkom z
$M_{22}$, označme ho $m_{22}$, potom je $m_{22}$ spojený aj s tým prvkom
z dvojice $m_{11}$,$m_{12}$, s ktorým nie je spojený $m_{21}$.
Prvok $m_{22}$ neexistuje iba vtedy, ak $|M_{22}|<|M_{21}|$. 
Ak $|M_{22}|>|M_{21}|$ potom v $M_{22}$ majú všetky vrcholy z $M_{21}$
izomorfný obraz a teda aj príslušná časť z párnej množiny.
Z horeuvedeného vyplýva, že menšia množina z prvého delenia má obraz,
izomorfný a spĺňajúci podmienky vety \ref{izomorfne}, 
vo väčšej množine vzniknutej z tohto delenia a keďže 
medzi vzorom a obrazom vedie hrana, veta platí.
\end{proof}

\begin{lem}
Maximálny počet hrán, ktoré môžu viesť v podgrafe hyperkocky indukovanom m
bodmi je daný postupnosťou $a_{m}$, kde $a_{1} = 0; a_{2} = 1;$ a pre 
$\forall i < 2; a_{i} = \lfloor \frac {i}{2} \rfloor + a_{\lfloor \frac{i}{2}
\rfloor}
+ a_{\lceil \frac {i}{2} \rceil } $
\end{lem}
\begin{proof}
V dôkaze použijeme úplnú indukciu. Báza indukcie je očividná: hyperkocka
nemá slučky a teda v jej jenovrcholovom indukovanom grafe nevedie 
hrana a hyperkocka nemá
násobné hrany, teda v jej indukovanom dvojvrcholovom podgrafe 
môže viesť najviac jedna hrana. Predpokladáme, že pre všetky členy s
indexom menším ako m platí, že udávajú najväčší možný počet hrán v
indukovanom podgrafe hyperkocky zo všetkých množín jej bodov, ktorých 
mohutnosť je rovná  indexu daného člena postupnosti. ďalej predpokladáme, že
pre každého člena postupnosti s menším indexom platí tvrdenie vety.
Ukážeme, že existuje indukovaný podgraf hyperkocky s m bodmi 
s najvyšším možným počtom hrán G taký, ktorý niektorá dimenzia delí na
polovice $G_{1}$ a $G_{2}$, teda na množiny vrcholov, ktorých mohutnosti sa líšia najviac o
jedna. To vyplýva z toho, že neexistuje taký podgraf hyperkocky, 
ktorý by delila inak a mal by zároveň vyšší počet hrán.
Nech taký podgraf $G^{,}$ existuje. Potom ho vybraná dimenzia delí na podgrafy s
počtom bodov $a_{i}$, $a_{j}$. Keďže sú obe neprázdne, sú menšie ako m a teda
pre ne platí indukčný predpoklad. $a_{i} = \lfloor \frac {i}{2} \rfloor + 
a_{\lfloor \frac{i}{2} \rfloor}+ a_{\lceil \frac {i}{2} \rceil }$ a $a_{j} = 
\lfloor \frac {j}{2} \rfloor + 
a_{\lfloor \frac{j}{2} \rfloor}+ a_{\lceil \frac {j}{2} \rceil }$
Bez ujmy na všeobecnosti, nech i < j. Aby sme zjednodušili indexovanie
označme $a_{\lfloor \frac{i}{2} \rfloor} = a_{i1}; a_{\lceil \frac {i}{2} \rceil }
= a_{i2}; a_{\lfloor \frac{j}{2} \rfloor} = a_{j1}$ a $a_{\lceil \frac {j}{2} \rceil }
= a_{j2}$.
Je zrejmé, že $a_{i2} - a_{i1} <= 1$ a $a_{j2} - a_{j1} <= 1$. Ak teda
rozdelíme G podľa dimenzie, ktorá ho delí na $G_{1}$ a $G_{2}$, ich mohutnosti
budú bez ujmy na všeobecnosti $a_{i2} + a_{j1}; a_{j2} + a_{i1}$. Z IP
zároveň platí, že $G_{1}$ a $G_{2}$ majú aspoň tak dobré rozdelenie ako na
množiny veľkostí $a_{i2}, a_{j1}$ respektíve $a_{j2}, a_{i1}$.
Počet hrán, ktoré vedú vo podgrafe $G^{,}$ je teda
$a_{m^{,}}=i + \lfloor \frac{i}{2} \rfloor 
+ \lfloor \frac{j}{2} \rfloor + a_{i1} + a_{i2} + a_{j1} + a_{j2}$.
Je zrejmé, že menšia z množín na ktoré sa rozdelí G bude mať veľkosť
$\lfloor \frac{m}{2} \rfloor$ a v týchto podmnožinách vedie aspoň $a_{i1} +
a_{j2} + \lfloor \frac{i}{2} \rfloor $ respektíve $a_{j1} + a_{i2} + 
\lceil \frac{i}{2} \rceil$ hrán.
Čiže $a_{m} >= \lfloor \frac{m}{2} \rfloor + a_{i1} +
a_{j2} + \lfloor \frac{i}{2} \rfloor + a_{j1} + a_{i2} + 
\lceil \frac{i}{2} \rceil$
Z toho si ľahko ukážeme, že $a_{m} >= a_{m^{,}}$.
 $a_{m} - a_{m^{,}} >= \lfloor \frac{m}{2}\rfloor + \lceil \frac{i}{2}
\rceil - i - \lfloor \frac{j}{2} \rfloor$ a keďže 
$\lfloor \frac{m}{2}\rfloor = \lfloor \frac{i+j}{2}\rfloor >= \lfloor \frac{i}{2} \rfloor + \lfloor \frac{j}{2} \rfloor$
potom platí, že $a_{m} - a_{m^{,}} >=0$. Tým sme dokázali, že neexistuje
lepšie rozdelenie podgrafu hyperkocky indukovaného m vrcholmi ako rozdelenie
na polovice. Z predošlých viet vyplýva, že takéto rozdelenie je možné.
\end{proof}




\begin{proof}
Zoberme jednu takúto dimenziu a rozdeľme body na dve množiny, podľa toho,
či v nej majú 1 alebo 0. Platia dve veci: medzi týmito dvomi množinami vedie
najviac toľko hrán, koľko má menšia z nich vrcholov; v iných dimenziách už
medzi týmito vrcholmi hrany nevedú.
Prvá vec vyplýva z toho, že každý vrchol má v jednej dimenzii najviac jednu
hranu (pri nekompletnej hyperkocke, pri kompletnej práve jednu).
Druhá vyplýva z toho, že hrana spája vrcholy, ktoré sa líšia práve v jednej
súradnici a vedie v dimenzii tejto súradnice. Ak by mala viesť medzi týmito
dvomi množinami hrana v nejakej inej súradnici, musela by spájať vrcholy,
ktoré sa líšia v najmenej dvoch súradniciach (v tej ktorá rozdeľuje vrcholy
na dve množiny a v tej v ktorej vedie táto hrana), čo nemôže.

Na tieto dve množiny sa teda môžeme pozrieť na každú zvlášť a znova ich
deliť podľa ďalších súradníc až kým sa nerozpadnú na jednovrcholové množiny.
Najväčší možný počet hrán vedúcich v tejto množine vrcholov je ohraničený
zhora sumov veľkosti menších množín cez všetky takéto delenia.

Toto ohraničenie získame spätne. Na konci máme m izolovaných vrcholov. Ako
mohli byť pospájané tak, aby medzi nimi viedol najväčší možný počet hrán?
Začnime s m izolovanými vrcholmi. V každom momente je jasné, že nech sa
najmenšia množina "odrezala" od ľubovoľnej inej, rátal sa ako maximálny
počet hrán mohutnosť tejto množiny.
Počet delení množín je konštantný, na čo sa dá ľahko použiť klasický príklad
s lámaním čokolády na m kúskov. Horné ohraničenie počtu hrán vedúcich v
množine bodov získame ako súčet miním cez všetky delenia. Pozrime sa na dve
najmenšie množiny v ľubovoľnej fáze delenia a pozrime sa na ne odspodu, tj.
pri spätnej konštrukcii delenia z konečných jednovrcholových množín - teda
na ich postupné spájanie.
Tvrdíme, že ľubovoľné spájanie nie je lepšie ako spojenie dvoch najmenších
množín. Sporom, nech existuje také najlepšie spájanie, ktoré nespája dve najmenšie
množiny a súčet jeho miním je ostro väčší ako takého, ktoré ich spája.

- TODO lema
\begin{lem}
Majme množinu množín vrcholov, označme
ju M. Nech existuje taká postupnosť spájania týchto množín,
ktorého súčet bude najvyšší možný. Nech sa v ňom nenachádza spojenie
najmenšej množiny s inou v prvom kroku, ukážeme že existuje poradie, kde sa
nachádza o krok skôr, ktoré je aspoň také dobré alebo pôvodné nie je
najlepšie.
\end{lem}
\begin{proof}
Ak sa najmenší prvok spája s množinou, ktorá bola v M, vieme spojenie minima
s ňou presunúť do úplne prvého kroku spájania.
Ak sa spája s množinou, ktorá vznikla spojením dvoch množín, premiestnime
krok pripojenia minima hneď za krok pri ktorom táto množina vznikla. Označme
  množiny z ktorých táto množina vznikla A,B a mohutnosť minima 
označme C, |A| = a; |B| = b; |C| = c.
Bez ujmy na všeobecnosti a =< b. Spájanie nebolo optimálne, lebo v jeho súčte sa nachádza
c+a, keby sa však spojila najprv množina A s C tak by bol celkový súčet
väčší, lebo by obsahoval buď c+b alebo c+(a+c).

\end{proof}






Každý vrchol je 0-rozmernou kockou sám o sebe. Všetky hrany, ktoré medzi
nimi môžu viesť v prvej dimenzii, v druhej, tretej atď. ... TODO
výde z toho hyperkocka a elegantnejšie ako predtým :) -- stále sa mi to
mezdá dokázané, že iný výber vrcholov nedá viac hrán ...

Treba maximalizovať počet dvojíc líšiacich sa v práve jednej dimenzii ...
V každej dimenzii má vrchol max jednu hranu

To že medzi vrcholmi vedie hrana znamená, že sú spolu v nejakej hyperkocke.
V čím väčšej sú, tým viac hrán z nich ide.

Ukážeme že maximalizácia n-rozmerných kociek maximalizuje počet n-1
rozmerných.
Vezmeme najväčšiu maximálnu disjunktnú množinu n-1 rozmerných kociek. Aké je
poprepájanie týchto kociek hranami pri ktorom je najviac n-1 rozmerných
kociek? 

Ak vedie hrana medzi vrcholmi a tieto sú spojené každý
hranami rovnakej dimenzie s inými dvomi vrcholmi, vedie hrana aj medzi
týmito dvomi vrcholmi. Je to dané tým, že ak odstránime dimenziu v ktorej
vedie hrana, tieto vrcholy sa zobrazia na seba.

Kocka vyššej dimenzie vznikne, ak spojíme dve n-1 rozmerné kocky
hranami podľa nejakého izomorfného zobrazenia dostaneme n-rozmernú kocku,
ktorá má $ {n choose {n-1}}$ n-1 rozmerných kociek.

\end{proof}

Izolované vrcholy -> jednorozmerné kocky -> dvojrozmerné kocky -> ... vždy
dve spojíme na jednu s vyšším rozmerom ...



Čiže trade-off medzi počtom značiek a pohybov ... stačí jedna návšteva
vrcholu ... otvorí pri nej všetky porty ... počet značiek = počet mimokostrových
(portových) hrán ... 

Čím vyššia dimenzia kocky, tým je menej kostrových hrán oproti portovým ->
stačí menej pohybov na požitie daného množstva značiek.

Ak vrcholov nie je presne toľko, ako v nejakej n-rozmernej kocke tj. ${2}^{n}$
tak ich vyberáme z vrcholov najmenšej kocky s väčším počtom vrcholov tak, že
vezmeme vrcholy nejakej (n-1) rozmernej kocky, ktorá je jej podgrafom, v
ďalšom kroku zvyšok vrcholov vyberieme z ešte nepoužitých tak, aby tvorili kocku
čo najväčšej dimenzie a tak ďalej, kým nevyberieme všetky vrcholy. Každá
n-rozmerná kocka sa dá rozdeliť na dve disjunktné (n-1)-rozmerné kocky.
Každý vrchol takejto kocky je spojený s práve jedným vrcholom z druhej
kocky. Agent teda vždy nájde kocku ktorá je najmenšia taká, že má viac
vrcholov. Jednu z týchto kociek pokryje a zvyšné vrcholy umiestni do druhej
kocky rekurzívnym spôsobom.

Týmto sme vyriešili hľadanie postupnosti k vrcholov s nyjväčším počtom hrán
medzi nimi. Teraz treba zistiť, ako veľká musí byť množina značiek potrebná
na konštrukciu týchto hrán.
Ak má agent minúť čo najmenej značiek pri tvorbe portových hrán, treba aby
ostali porty otvorené čo najkratšie - čas meriame pohybmi agenta.

Počet značiek, ktroé agent potrebuje pre konštrukciu n-rozmernej kocky je
suma dva na i minus jedna pre i od jedna po n. (po konštrukcii jednej kocky
ostávajú otorené porty do druhej a agent potrebuje rovnaký počet značiek ako
na predošlú ... - lenže postupne sa uvoľnujú) pozn. bude to teda asi \#
portov pre štvrtinovú kocku + dva na i munus jedna

\subsection{2-rozmerná kocka}
Elementárny prípad. Stačí jedna portová značka a tri pohyby - operácue NH.
\subsection{3-rozmerná kocka}
\paragraph{1 značka}
Pri použití jednej zančky na porty vie agent trojrozmernú kocku zostrojiť na 15 pohybov:
OP, NH, NH, NH, OP, OP, S1, NH, NH, OP, OP, S2, S2, NH, OP, OP, S1, S1, NH, OP, OP, S1, S2, S3, OP. Na
menej sa to nedá. Agent potrebuje skonštruovať 5 mimokostrových hrán a na
každú potrebuje aspoň 3 pohyby.

\paragraph{2 značky}
Trojrozmernú kocku pri použití dvoch značiek na porty dokáže agent zostrojiť
na 11 pohybov a menej sa nedá. Operácie agenta:
OP1, NH, NH, OP2, NH, OP1, NH, OP1, NH, OP2, NH, OP2, NH, OP1, OP1, S1, S1, S2, OP1, S1, OP2
Na menej sa nedá, pretože porty treba otvoriť medzi dvojicou vrcholov
líšiacich sa v práve jednej súradnici. Teda medzi otovrením portov jednej
hrany potrebuje agent aspoň dve preklopenia na rovnakej súradnici a jedno na
inej. Vie využiť najviac jedno preklopenie, ktoré nastalo pri konštrukcii
predošlej hrany, čiže na konštrukciu hrany potrebuje najmenej dva ďalšie
pohyby. Na konštrukciu prvej hrany potrebuje tri pohyby. $3 + 2 * 8 = 11$

\paragraph{3 značky}
Trojrozmernú kocku agent za použitia troch portových značiek zostrojí na 9
pohybov. Na menej sa nedá.
Operácie agenta: OP1, OP3, NH, NH, OP2, NH, OP1, NH, OP1, NH, OP2, NH, OP2,
NH, OP1, OP3, S3(?), S1, OP2
S použitím menšieho množstva pohybov to agent urobiť nemôže. Kým vznikne
prvá portová hrana - značka sa uvoľní na ďalšie použitie,vytvorí agent tri
nové vrcholy. Teda je vo štvrtom vrchole. Súčet  stupňov predošlých troch je
deväť, z čoho päť zaberú nové hrany. To znamená, že do jedného z týchto
vrcholov sa agent bude musieť vrátiť otvoriť port alebo z neho vytvoriť novú
hranu. Keby tento vrchol bol tretí podľa vzniku, tak sa doň agent môže
vrátiť na jeden ťah. Tu môžu nastať dve možnosti: pôjde tam otvoriť port
alebo vytvoriť novú hranu. Ak otvoriť port, bude mu chýbať značka na štvrtom
vrchole podľa vzniku a bude sa musieť vrátiť aj do neho. Ak otvorí port s
novouvoľnenou značkou na štvrtom vrchole a z tretieho urobí NH. Ocitne sa vo
vrchole v ktorom nemôže otvoriť port, lebo sa nelíši v práve jednej
súradnici od žiadneho z vrcholov v ktorých sú porty pootvárané. Teda v oboch
prípadoch potrebujem ďalšie dva pohyby, ktoré nie sú operáciou NH. Ak sa
agent nevráti hneď do vrchola, v ktorom neotvoril port, bude musieť prejsť
do niektorého z jeho dvoch susedov (lebo z nich už odišiel a z iného vrchola
sa tam nedostane). To sú dva pohyby okrem operácií NH. Počet operácií NH je
konštantný pre daný  graf a rovná sa počtu hrán kostry. V tomto prípade je
to sedem. Teda agent vybavený tromi rôznymi značkami na porty potrebuje
aspoň deväť krokov na konštrukciu trojrozmernej kocky.

\paragraph{4 značky}
4 rôzne značky portov agentovi na konštrukciu trojrozmernej kocky stačia:
OP1, OP2, NH, OP3, NH, OP4, NH, OP1, NH, OP1, NH, OP4, NH, OP3, NH, OP2, OP1
Agent vzkoná sedeme pohybov, čo je nevyhnutné minimum na konštrukciu
trojrozmernej kocky. Pridanie ďalších značiek už teda nič nevylepší.

\subsection{4-rozmerná kocka}
\paragraph{1 značka}
Ak chcem skonštruovať portové hrany použitím troch pohybov na hranu a žiadne
ďalšie pohyby, tak musím vo vrchole, kam otvorením portu vznikne nová
portová hrana, otvoriť port znovu. Portové hrany teda budú tvoriť súvislý
ťah. 



Všetkých hrán v štvorrozmernej kocke je 32, pričom kostrových je 15.
Portových hrán je teda 17. Sled portových hrán sa nemôže pretínať, pretože
nemôže existovať vrchol do ktorého nevedie kostrová hrana a každý vrchol má
stupeň 4. Keďže stupeň vrcholov je párny, kostru treba zostrojiť tak, aby
dva z nich mali nepárny stupeň a zvyšné párny. Čo je nutná podmienka zostrojenia
jediného ťahu zo všetkých zvyšných hrán. Kostra má aposň dva listy a stupeň
listov je nepárny. Ak teda kostra nemá mať väčší počet listov, musia mať
zvyšné vrcholy stupeň dva. Kostra je teda cesta. Ukážeme, že nie je možné
zostrojiť štvorrozmernú kocku na $3 * 17 = 51$ pohybov. Vrcholy v kotorých kostra-cesta začína
majú tri portové hrany. Aby sme použili len tri pohyby na jednu portovú
hranu, potrebujeme začať portové hrany konštruovať v jednom z týchto
vrcholov. v týchto vrcholoch začína aj cesta-kostra. Keďže kostra vzniká
tak, že na už existujúcom vrchole vykoná agent operáciu NH, v každom momente
je práve jeden vrchol v ktorom môže byť agent, ak má skonštruovať graf na 51
pohybov. Tým však, že agent odišiel z prvého vrchola a tomu chýbajú ešte dve
hrany, sa doň bude musieť vrátiť a otvoriť v ňom port. V konštruovanom grafe
(medzivýsledku) sa však agent od prvého vrchola vzďaľuje každou portovou hranou.
Pri skonštruovaní posledného vrchola bude teda od prvého v medzivýsledku
konštrukcie ďalej ako tri a teda nie je schopný skonštruovať graf týmto
spôsobom. 
Vrchol do ktorého ide agent konštruovať portovú hranu musí byť v
medzivýsledku vzdialený práve tri od druhého vrchola s otvoreným portom a
súčsne sa musí líšiť v práve jednej dimenzii. Pri tomto postupe žiadny
takýto vrchol k poslednému skonštruovanému vrcholu v momente otvorenia portu nie je
a jediný, ktorý sa tak dá vytvoriť je pomocou prechodu po novej portovej
hrane a dvoch opráciách NH, alebo po troch operáciách NH.
Ak použije prvý spôsob, kým sa dá, dostane sa do bodu z ktorého sa už
tento nedá uplatniť a už nebude schopný skonštruovať ďalšiu hranu na tri
pohyby.
Pri použití druhého spôsobu z inicializačného vrchola trikrát sa dostane
agent do situácie, kefy je od inicializačného vrchola vzdialený na
medzivýsledku o tri hrany a zároveň sa ich polohy líšia v práve jednej
dimenzii. Takže agent môže skonštruovať zvyšné dve hrany pre inicializačný
vrchol.
(TODO mohol by to urobiť aj neskôr, ale nejde to, lebo štvorrozmerná kocka má len
15 vrcholov a už by mu nevydalo vytvoriť tri portové hrany pomocou operácií
NH)
Keď má agent z vrcholu vytvoriť portovú hranu a odchádza z neho operáciou
NH, odíde po jednej dimenzii a portová hrana sa otvorí v druhej. Teda
vytvorí hrany v dvoch dimenziách vrcholu.
Po skonštruovaní prvej portovej hrany má vrchol v ktorom agent skončí ešte
"voľné" dve dimenzie. Z jednej má ísť portová hrana a z druhej kostrová.
Teda po jednej agent odíde a v druhej sa po troch pohyboch bude líšiť.
Rovnakáá situácia nastane po vytvorení druhej portovej hrany. Tretia portová
hrana musí mať rovnakú dimenziu ako prvá, aby sa agent po jej konštrukcii
líšil s inicializačným vrcholom v práve jednej dimenzii. štvrtá portová
hrana je teda tiež daná. Dimenzie sú navzájom zameniteľné, takže na ich
výbere nezáleží.
(TODO Začiatok konštrukcie štvorrozmernej kocky je deterministický - možno
je celá konštrukcia ...)

\iffalse
TODO:
1, pozrieť, čo všetko sa dá nájsť o hyperkocke a mriežke ..

%tak toto sa ukazalo ako blbost :)
Keďže tento spôsob ako jediný pripadal do úvahy na 51 ťahov a
nefunguje, na komštrukciu štvorrozmernej kocky treba aspoň 52 ťahov pri
jedinej portovej značke.
Je jasné, že agent sa do prvého vrchola bude musieť vrátiť, lebo po skonštruovaní
prvej portovej hrany mu ešte dve hrany chýbajú. Pri 52 pohyboch môže na
konštrukciu jednej hrany minúť 4 pohyby a na zvyšné si musí opäť vystačiť s
tromi. Teda portové hrany môžu tvoriť dva ťahy, ktorých dva konce sú
vzdialené o jednu hranu alebo jeden ťah. Teda kostra môže mať tri listy a
vrchol stupňa tri i štyri listy a vrchol stupňa štyri. Zvyšné vrcholy kostry
majú stupeň dva. Teda kostra sa môže rozvetvovať na najviac jednom mieste.
Po konštrukcii prvej hrany 
\fi
