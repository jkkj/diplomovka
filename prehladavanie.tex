\chapter{Prehľadávanie grafu}
\thispagestyle{empty}
Jednou zo skúmaných otázok o našom prístupe ku konštrukcii grafu je otázka či a v akej miere sa jedná o duálny problém k prehľadávaniu grafu. V tejto kapitole uvádzame čitateľa do problematiky prehľadávania grafu, príslušných definícií a pre nás zaujímavých výsledkov.
\section{Tu bude Úvod}
<TODO> Zatiaľ je tu všetok prehľad, potom to porozhadzujem.</TODO>\\
Ako uvádza /TODO cit. Online Graph Exploration ... Nicole Megow1, Kurt Mehlhorn1, and Pascal Schweitzer2 .../
v problémoch prehľadávania má agent zostrojiť kompletnú mapu grafu (prostredia) bez akejkoľvek počiatočnej a priori informácii o topológii.
Tento článok sa zaoberá problémom prehľadávania na neorientovaných spojitých grafoch s hranami ováhovanými nezápornými reálnymi číslami a označenými (labeled) vrcholmi (agent ich vie rozlíšiť).
Efektivita online algoritmov sa zisťuje ich porovnávaním s  prislúchajúcimi offline riešeniami. V tomto prípade je prislúchajúcim offline problémom TSP.

/TODO pozrieť zdroje článku/ Medzi súvisiacimi prácami spomína: prvý formálny model na prehľadávanie grafu [27].
Prvý raz bolo skúmané prehľadávanie celého grafu v [10] avšak uvažovali preskúmanie každej hrany v označenom orientovanom grafe. Teda, keď agent prišiel do vrcholu, dostal počet neobjavených hrán odchádzajúcich z vrcholu ale už nie kam vedú.
Zodpovedajúcim offline problémom pre tento problém však nie je TSP ale polvnomiálne riešiteľný Chinese postman problem.
(/TODO/ Takéto zadanie má bližšie aj k nášmu prvotnému poňatiu problému konštrukcie grafu pomocou jedného mobilného agenta)
(/TODO/ je spomínané intenzívne štúdium na orientovaných aj neorientovaných grafoch - pozrieť)
/TODO/ Je aj iná trieda problémov, v literatúre označovaných ako online TSP, kde graf je známy dopredu a vrcholy, ktoré majú byť navštívené sa objavujú postupne. Zodpovedajúci offline problém pre túto triedu problémov je TSP s dátumami uvedenia (release dates).
\newline

Ďalši článok /TODO citácia Graph exploration by a finite automaton ... Pierre Fraigniauda,1, David Ilcinkasa .../
skúma prehľadávanie grafu konečným automatom (robotom). Graf je neoznačený s hranami lokálne označenými v každom uzle.
(TODO čo sa ešte viac približuje duálnemu problému nášho poňatia konštrukcie grafu)
Úlohou robota je prehľadať každú hranu v grafe bez akejkoľvek predošlej znalosti o veľkosti a topológii grafu. Medzi zaujímavé výsledky tohto článku patrí, že existujú planárne grafy, pre ktoré sa to nedá.
V prípade unikátneho značenia hrán a vrcholov môže byť prehľadanie dosiahnuté ľahko. Existujú neznáme prostredia, v ktorých takéto značenie nemusí byť k dispozícii, alebo robot nemusí byť schopný rozlíšiť od seba dva podobné lable. Z tohto dôvodu sa skúma prehľadávanie anonymných grafov; tj. grafov bez unikátne označených hrán a vrcholov.
Je len samozrejmé, že bez možnosti rozlíšiť lokálne porty by nebolo možné prehľadať dokonca ani hviezdu s tromi listami.
Predpokladá sa teda v každom vrchole označenie portov 1 .. d, kde d je stupeň vrchola. Avšak nepožaduje sa žiadna konzistencia tohto značenia. Keďže v mnohých aplikáciách sa požaduje, aby bol robot malé lacné zariadenie, autori článku optimalizujú pamäť. Takže hľadajú robota schopného preskúmať graf danej neznámej veľkosti s tak malou pamäťou, ako je to len možné. Robot s k-bitovou konečnou pamäťou sa modeluje konečným automatom. Je dokázané /TODO/, že konečný automat s konečným počtom kamienkov nedokáže prehľadať každý graf.
\newline
Labyrint je šachovnica /TODO/ $Z^2$ so zakázanými políčkami. Labyrint je konečný. Prehľadávaním labyrintov sa vo veľkej miere zaoberal 
Budach /TODO/.
Prehľadať konečný labyrint znamená, že robot je schopný odísť ľubovoľne ďaleko zo svojej štartovnej pozície pre každú štartovnú pozíciu. Hrany labyrintu sú konzistentne označené svetovými stranami (východ,juh,západ,sever).
Rollik [TODO ref. 45] dokázal, že žiadna konečná množina konečných automatov nedokáže prehľadať všetky kubické planárne grafy.

Výsledkom článku /TODO ... Pierre Fraigniauda .../ je veľmi blízke ohraničenie trapu pre K-stavový automat. Trap je graf, ktorý nie je možné celý prehľadať uatomatom bez kamienkov (pebbles). Výsledok článku hovorí, že pre každý K-stavový automat a $d\geq 3$ existuje K+1 vrcholový graf s max. stupňom d, ktorý sa nedá prehľadať týmto automatom.
\section{Definície}
\section{Výsledky}