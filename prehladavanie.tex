\chapter{Prehľadávanie grafu}
\thispagestyle{empty}
\section{Úvod}
Jednou z motivácií nášho prístupu ku konštrukcii grafu je možnosť využitia
takejto konštrukcie ako duálneho problému k prehľadávaniu grafu. Náš model
konštrukcie vychádza z modelov používaných pri prehľadávaní grafov s tým
rozdielom, že kým v niektorých úlohách z prehľadávania je daný neznámy graf
a úlohov agenta je skonštruovať jeho mapu, v našom modely je dané ako má
graf vyzerať a úlohou agenta je skonštruovať ho pomocou operácií, ktoré má k
dispozícii.
V tejto kapitole uvádzame čitateľa do problematiky prehľadávania grafu, 
príslušných definícií a zaujímavých výsledkov z tejto oblasti. Nájde tu
prehľad vybraných prístupov k tejto problematike a používaných modelov.

V problémoch prehľadávania má agent zostrojiť kompletnú mapu grafu 
(prostredia) bez akejkoľvek počiatočnej informácie o jeho topológii,
prípadne len navštíviť každý vrchol alebo prejsť každou hranou.
Tiež sú známe tri typy riešených úloh podľa toho, ako agent ukončí
prehľadávanie. Prvým typom je prehľadávanie s návratom, keď agent po
skončení musí zastať v počiatočnom vrchole. Druhým typom je prehľadávanie so
zastavením, kedy agent musí v konečnom čase po prehľadaní celého grafu
zastaviť. Posledným, tretím, typom úloh o prehľadávaní grafu je trvalé
prehľadávanie. V tomto prípade agent pokračuje v prechádzaní grafu aj po
jeho úplnom prehľadaní. Patria sem tiež úlohy, v ktorých má agent
navštevovať každý daný prvok grafu (vrchol alebo hranu) s určitou periódou.
Význam takejto úlohy je napríklad pri neustálej kontrole siete, ktorá sa
môže kaziť počas prevádzky.
Úspešnosť algoritmov prehľadávania grafov sa porovnáva s takzvanými offline
problémami. Pri offline probléme agent graf pozná a jeho úlohou je navštíviť
každý jeho vrchol, prípadne hranu podľa rovnakého zadania ako pri
prehľadávaní. Napríklad ak má agent navštíviť každý vrchol v neznámom
ohodnotenom grafe, online problémom je problém obchodného cestujúceho (TSP).

Prvý formálny model na prehľadávanie grafu uviedol \cite{SPWM}.
Prvý raz bolo skúmané prehľadávanie celého grafu v \cite{EUG}, 
avšak uvažovali preskúmanie každej hrany v označenom orientovanom grafe. 
Teda, keď agent prišiel do vrcholu, dostal informáciu o počte neobjavených
hrán odchádzajúcich z vrcholu ale už nie informáciu, kam tieto hrany vedú.
Zodpovedajúcim offline problémom pre tento problém však nie je TSP 
ale polynomiálne riešiteľný Problém čínskeho poštára.
Takéto zadanie problému prehľadávania má blízko k nášmu poňatiu 
problému konštrukcie grafu pomocou jedného mobilného agenta. V tejto oblasti
prebieha rozsiahli výskum na orientovaných aj neorientovaných grafoch.
Existuje aj iná trieda problémov, v literatúre označovaných ako online TSP, 
kde graf je známy dopredu a vrcholy, ktoré majú byť navštívené sa o
bjavujú postupne. 
Zodpovedajúci offline problém pre túto triedu problémov je TSP s dátumami
uvoľnenia (release dates).

Predpoklad, že konečný automat s konečným počtom 
kamienkov nedokáže prehľadať každý graf bol dlho otvoreným problémom až
\cite{ROLL} dokázal o niečo širšie tvrdenie. Rollik \cite{ROLL} dokázal, 
že žiadna konečná množina konečných automatov nedokáže prehľadať všetky 
kubické planárne grafy.

Labyrint je šachovnica $Z^2$ so zakázanými políčkami. 
Labyrint je konečný. Prehľadávaním labyrintov sa vo veľkej miere zaoberal 
Budach.
Prehľadať konečný labyrint znamená, že robot je schopný odísť ľubovoľne 
ďaleko zo svojej štartovnej pozície pre každú štartovnú pozíciu. 
Hrany labyrintu sú konzistentne označené svetovými stranami (východ, juh, 
západ, sever).

\section{Prehľadávanie jedným agentom}
\cite{OGE} sa zaoberá problémom prehľadávania na neorientovaných 
spojitých grafoch s hranami ováhovanými nezápornými reálnymi číslami a 
označenými (labeled) vrcholmi (agent ich vie rozlíšiť).
Efektivita online algoritmov sa zisťuje ich porovnávaním s  prislúchajúcimi 
offline riešeniami. V tomto prípade je prislúchajúcim offline problémom
Problém obchodného cestujúceho (TSP).

Ďalši článok \cite{GEFA}
skúma prehľadávanie grafu konečným automatom (robotom). 
Graf je neoznačený s hranami lokálne označenými v každom uzle; rovnako ako v
našom modely konštrukcie.
Úlohou robota je prehľadať každú hranu v grafe bez akejkoľvek predošlej 
znalosti o veľkosti a topológii grafu. Medzi zaujímavé výsledky tohto článku 
patrí dôkaz existencie planárnych grafov, pre ktoré sa to nedá.
V prípade unikátneho značenia hrán a vrcholov môže byť prehľadanie 
dosiahnuté ľahko. Existujú neznáme prostredia, v ktorých takéto značenie 
nemusí byť k dispozícii, alebo robot nemusí byť schopný rozlíšiť od seba dve 
podobné značky. Z tohto dôvodu sa skúma prehľadávanie anonymných grafov; 
tj. grafov bez unikátne označených hrán a vrcholov.
Je len samozrejmé, že bez možnosti lokálne rozlíšiť konce hrán by nebolo možné 
prehľadať dokonca ani hviezdu s tromi listami.
Predpokladá sa teda v každom vrchole označenie portov $1 .. d$, kde d je 
stupeň vrchola. Avšak nepožaduje sa žiadna konzistencia tohto značenia. 
Keďže v mnohých aplikáciách sa požaduje, aby bol robot malé lacné zariadenie, 
autori článku optimalizujú pamäť. Takže hľadajú robota schopného preskúmať 
graf danej neznámej veľkosti s tak malou pamäťou, ako je to len možné. 
Robot s k-bitovou konečnou pamäťou sa modeluje konečným automatom. 
Trap je graf, ktorý nie je možné 
celý prehľadať automatom bez kamienkov (pebbles). Výsledok článku hovorí, 
že pre každý K-stavový automat a $d\geq 3$ existuje K+1 vrcholový graf - trap
s max. stupňom d, ktorý sa nedá prehľadať týmto automatom.

\section{Distribuované prehľadávanie}
Autori článku \cite{DEA} používajú pre prehľadávanie neznámeho prostredia
viacero totožných agentov. Neznáme prostredie je v tomto prípade opäť
modelované grafom a úlohou agentov je skonštruovať totožnú mapu grafu pre
každého agenta. Bolo dosiahnuté deterministické riešenie za čo najslabších možných
nastavení, pričom agenty \footnote{Na návrh staršieho kolegu skloňujeme
slovo agent v našom kontexte ako neživotné} poznali iba veľkosť grafu alebo
počet agentov.

Autori skúmajú distribuovanú verziu
klasického problému prehľadania garfu s návratom do počiatočného vrchola,
pričom agent má prejsť všetky hrany neoznačeného (anonymného) grafu. Počas
tohto procesu je úlohou agenta vytvoriť mapu grafu. Model pozostáva z
neoznačeného grafu a $k$ agentov roztrúsených po $n$ vrcholoch grafu.
Komunikácia medzi agentmi prebieha prostredníctvom zápisov na malé tabuľky
nachádzajúce sa v každom vrchole grafu. Tento problém je zložitejší od
prehľadania grafu jediným agentom, pretože vyžaduje kooperáciu medzi
agentmi. Keďže je tento problém vo všeobecnosti deterministicky
neriešiteľný, článok predstavuje deterministický algoritmus za podmienky, že
$n$ a $k$ sú nesúdeliteľné.

Vo svojej práci dávajú autori
svoj výskum do súvisu s ostatnými problémami distribuovaných výpočtov, ako
je napríklad voľba šéfa.

\section{Prehľadávanie s radou}
Z praxe sú známe situácie, keď graf, ktorý má agent prehľadať, nie je úplne
neznámy. Teda je o ňom dostupná nejaká informácia. 
