
\chapter{Grafové gramatiky}%alias "čo sa už skúšalo"
\thispagestyle{empty}
\section{Úvod}
Naša práca sa zaoberá vytváraním grafov a preto je nutné spomenúť 
najpoužívanejší formalizmus vytvárania grafov - grafové gramatiky.
Cieľom tejto kapitoly je uviesť čitateľa do problematiky grafových 
gramatík a poskytnúť mu stručný prehľad tejto oblasti.

Predpokladáme, že čitateľ má základné vedomosti z teórie formálnych 
jazykov a automatov.

\section{Hlavné prístupy}
Teória okolo grafových gramatík sa budovala podobne ako pri formálnych
jazykoch a automatoch z viacerých prístupov pri ktorých boli neskôr dokázané
určité ekvivalencie. Z toho vyplýva množstvo tried gramatík, prístupov a
definícií.
V konečnom dôsledku však možno rozlíšiť dva hlavné prístupy ku grafovým
gramatikám - algoritmický (spájací) a algebraický (lepiaci).
Pre ilustráciu uvedieme niekoľko tried a skúmaných problémov.
Grafové gramatiky sa týkajú našej práce len okrajovo, v prípade hlbšieho 
záujmu oporúčame čitateľovi publikáciu Handbook of Graph Grammars \cite{GG} ,
z ktorej sme aj my čerpali.

\section{Všeobecné pojmy}
Za účelom uvedenia čitateľa do problematiky grafových gramatík sme si
vybrali skupinu grafových gramatík založenú na prepisovaní vrcholov. Vo
všeobecnosti má pravidlo odvodenia v grafovej gramatike tvar $(M, D, E)$,
kde $M$ a $D$ sú grafy a $E$ je vkladací mechanizmus. Takéto odvodenie môže
byť použité na ľubovoľný graf $H$, ktorý obsahuje podgraf $M$. Jeden výskyt
podgrafu $M$ je nahradený grafom $D$ a spojený so zvyškom grafu $H$
mechanizmom popísaným $E$. Zvyšok grafu H označujeme $H^{-}$.
Na základe mechanizmu $E$ sa rozlišujú spomínané dva prístupy. Pri lepiacom
prístupe sa časti $H^{-}$ stotožnia s časťami $M$ a následne $D$. Graf $D$
je skrz ne následne "vlepený" do grafu $H^{-}$.
Pri spájacom prístupe vznikajú podľa $E$ nové "spájacie" hrany medzi $D$ a 
$H^{-}$.
V spájacom prístupe sa v bezkontextových gramatikách skladá graf $M$ len z
jedného vrcholu. Teda vrchol je nahrádzaný grafom. Požaduje sa tiež, aby
postupnosť nahrádzania vrcholov nemala vplyv na konečný výsledok. Pritom je
viacero prirodzených spôsobov definovania mechanizmu $E$ a obmedzení na
"pravé strany" pravidiel - graf $D$.


\section{NLC gramatiky}
NLC (Node Label Controlled) grafové gramatiky sú jedným 
z najjednoduchších mechanizmov prepisovania grafov. Pravidlá NLC gramatiky 
sú tvaru $X \to D$, kde X je značka (label) prepisovaného vrchola a D je 
dcérsky graf. Pre použitie pravidie v NLC neexistujú žiadne obmedzenia, teda 
pravidlo sa môže použiť na ľubovoľný vrchol so značkou X.
Vkladanie je v NLC lokálne, tj. vrcholy grafu D môžu byť spojené len s tými 
vrcholmi, s ktorými bol spojený prepísaný vrchol X.
O NLC gramatikách, kde odvodenie je nezávislé na poradí použitia pravidiel 
odvodenia, hovoríme, že majú Church-Rosserovu vlastnosť 
- sú konfluentné (confluent). 
Vo všeobecnosti NLC túto vlastnosť nemajú a pri použití rovnakej sady 
pravidiel môžu vzniknúť rôzne grafy. 

\begin{defin}
NLC grafová gramatika je 5-tica $G = (\Sigma ;\Delta ; P;C; S)$ kde
$\Sigma  - \Delta$  je abeceda neterminálnych znaciek uzlov a
 $\Delta  (\Delta  \subseteq  \Sigma )$ je abeceda
terminálnych znaciek uzlov. P je konecná množina NLC pravidiel, C je
binárna relácia spojení nad $\Sigma$  a S je pociatocný graf.
\end{defin}

\begin{defin}
Grafový jazyk generovaný G je $L(G) = \{H \in GR_\Delta  | S \Rightarrow ^*H\}
, kde GR_\Delta $  je množina neorientovaných grafov so značkami uzlov 
v $\Delta ,\Rightarrow $  reprezentuje jeden prepisovací krok, 
a $\Rightarrow ^* $ reprezentuje odvodenie, t.j. postupnosť prepisovacích 
krokov.
\end{defin}

\section{NCE gramatiky}
NCE (Neighborhood Controlled Embeding) grafové gramatiky sú rozšírením NLC 
grafových gramatík.V NCE mechanizme prepisovania je, 
na rozdiel od NLC, množina spájacích inštrukcií C definovaná 
pre každé pravidlo zvlášť a NCE gramatika môže k uzlom podgrafu D 
pristupovať priamo napr. ich očíslovaním; nemusí používať len značky týchto
vrcholov. Toto už nie je možné pri vrcholoch hosťovského grafu (graf $H$),
pretože v ňom môže byť počet susedov nahrádzaného vrchola neohraničený.
Existujú rôzne rozšírenia NCE gramatík.
edNCE gramatiky sú rozšírením NCE gramatík s dynamickým 
preznačovaním hrán pri prepisovaní. 
Písmenko e znamená, že sa  budú značiť hrany a písmenko d označuje 
použitie orientovaných grafov.

\begin{defin}
edNCE grafová gramatika je definovaná ako 6-tica 
$G =(\Sigma ;\Delta ;\Gamma ;\Omega ; P; S)$. 
Kde $\Sigma $  je abeceda značiek uzlov, 
$\Delta  \subseteq  \Sigma $ je abeceda terminálnych značiek uzlov, 
$\Gamma $ je abeceda značiek hrán, 
$\Omega \subseteq  \Sigma $  je abeceda terminálnych
značiek hrán, P je konečná množina pravidiel a $S \in  \Sigma - \Delta $ je
počiatočný neterminál. Pravidlá sú tvaru $X \to (D,C), kde X \in  \Sigma - \Delta $ a
$(D,C) \in  GRE_{\Sigma ,\Gamma }$. Relácia spojení je 
$C \subseteq  \Sigma \times \Gamma \times \Gamma V_H \times {in,out}$.
\end{defin}

Vo všeobecnosti množinu grafov izomorfných s grafom H budeme značit
$[ H ] $. Niekedy sa hovorí o H ako „konkrétnom“ grafe a o $[ H ]$ ako „abstraktnom“
grafe.

\begin{defin}
Množinu všetkých konkrétnych grafov nad $\Sigma$ a $\Gamma $ značíme
$GR_{\Sigma ,\Gamma }$ a množinu všetkých abstraktných grafov značíme 
$[GR_{\Sigma ,\Gamma }]$. $GRE_{\Sigma ,\Gamma }$ označuje množinu grafov 
s vnorením nad $\Sigma ,\Gamma $. Obyčajný graf môžeme považovať
za graf s prázdnym vnorením, teda $GR_{\Sigma ,\Gamma } 
\subseteq GRE_{\Sigma ,\Gamma }$.
\end{defin}

\begin{defin}
Grafový jazyk generovaný G je $L(G) = \{[H] | H \in GR_{\Sigma ,\Omega }$ 
a $sn(S,z) \Rightarrow ^* H pre nejaké z \} $
(t.j. všetky grafy v odvodení G, ktoré majú iba terminálne značky hrán aj uzlov).
\end{defin}

Teraz si definujeme hlavnú triedu bezkontextových grafových gramatík so
spájacím prístupom. Označujeme ju C-edNCE. Táto je najrozšírenejšia v
spájacom prístupe má značne všeobecný mechanizmus vkladania a žiadne
reštrikcie na pravé strany.

\begin{defin}
edNCE gramatika $G =(\Sigma ;\Delta ;\Gamma ;\Omega ; P; S)$ je konfluentná
(confluent), alebo C-edNCE gramatika, ak pre všetky pravidlá $X_1 \to (D1,C1)$
a $X_2 \to (D_2,C_2)$ z P, všetky uzly $x_1 \in V_{D_1} a x_2 \in V_{D_2}$ , a všetky hrany so
značkami $\alpha ,\delta \in \Gamma $, platia nasledujúce ekvivalencie:
$\exists \beta \in \Gamma  : (X_2, \alpha /\beta , x_1, out) \in C_1 a (\lambda _{D_1}(x_1), \beta /\delta , x_2, in) \in C_2
\Leftrightarrow 
( \exists \gamma \in \Gamma  : (X_1, \alpha /\gamma , x_2, in) \in C_2 a (\lambda _{D_2}(x_2),\gamma /\delta , x_1, out) \in C_1$.
\end{defin}

\begin{pozn}
Existuje charakterizácia C-edNCE nezávislá na gramatikách pomocou Monadic Second Order (MSO) logiky.
Veľa výsledkov ju pre jednoduchosť využíva v dôkazoch.
\end{pozn}

\section{Rozličné známe výsledky}

Hodno spomenúť, že vlastnosti, ktoré poznáme ako bezkontextové nemusia byť
nutne konfluentné. \cite{GG} spomína pomerne jednoduchú lemu o
konfluentnosti gramatiky, kde každý vrchol má "svoju" polovicu značky hrany a
pri odvodení mení len tú. Následne nezáleží na poradí odvodenia dvoch
vrcholov spojených hranou, lebo tieto menia len "svoju" časť značky hrany.
Medzi zaujímavé veci v teórii o C-edNCE gramatikách patrí existencia
takzvaných zakázaných (forbidden) hrán. Sú to hrany označené neterminálnym
symbolom spájajúce dvojice terminálnych vrcholov. Z grafu obsahujúceho čo
len jedinú takúto hranu sa už konečný graf nestane.
Existujú mnohé modifikácie gramatík, ktoré sme si bližšie nespomínali.
Skúmali sa aj problémy rozpoznávania, či graf patrí do jazyka; 
napríklad existuje NP-úplný LIN-A-edNCE jazyk.
Taktiež sú preskúmané rôzne normálne formy. Veta o medzere hovorí, že
existuje $c$ pre ktoré sa nedá odvodiť v danom jazyku dosť grafov a tak
ďalej.

\section{Grafové gramatiky s nahrádzaním hyperhrán}
Pre úplnosť venujeme krátky priestor aj vlepovaciemu prístupu v grafových
gramatikách.
Grafové gramatiky s nahrádzaním hyperhrán sú považované za grafovú
ekvivalenciu k reťazcovým gramatikám. Ich začiatky siahajú do skorých 70.
rokov. Základom je hyperhrana. Hyperhrana je atomický prvok s fixným počtom
chápadiel - typov. Môže byť pripojená k ľubovoľnej štruktúre odvodenej z
množiny vrcholov pripojením chápadla do vrcholu. Hyperhrana môže byť
prepísaná štruktúrou $R$. $R$ však musí mať, kvôli lepiacemu prístupu,
rovnaký počet externých vrcholov ako nahrádzaná hrana typov.
Gramatiky v tejto skupine majú konečný počet pravidiel a počiatočnú
štruktúru. Jazyk tvoria množiny konečných štruktúr odvodených takouto
gramatikou. Prepísanie hyperhrany je bezkontexotvé, pretože nezasahuje do
okolia. Bolo dokázaných niekoľko výsledkov, ktoré majú blízko k podobným
bezkontextovým gramatikám na reťazcoch. Ukázalo sa, že veľa zaujímavých
problémov je pre tieto gramatiky rozhodnuteľných.
Môžeme sa pýtať dva typy otázok: či hypergrafy generované príslušnou
gramatikou spĺňajú vybrané vlastnosti alebo klasickú otázku príslušnosti do
jazyka. Otázka príslušmosti je vo všeobecnosti NP-kompletná a len niektoré
obmedzené  triedy sú polynomiálne ťažké.
\iffalse
/? TODO
regulárne grafové gramatiky
TODO ?/

Zaujímajú nás predovšetkým výsledky týkajúce sa "náročnosti" generovania 
grafov, ktoré je možné porovnávať s výsledkami nášho prístupu.
\fi


