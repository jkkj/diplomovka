\chapter{Úvod}
\thispagestyle{empty}
Graf je vhodnou reprezentáciu údajov potrebných na riešenie mnohých problémov.
V tomto smere našiel široké uplatnenie aj v informatike a oblastiach 
matematiky, ktoré sú s ňou úzko späté.
My sa budeme zaoberať pojmom graf tak, ako je chápaný v Teórii grafov.
V našom ponímaní je graf abstraktnou štruktúrou obsahujúcou vrcholy a hrany.
Pričom každá hrana má práve dva konce - vrcholy, ktoré spája. Ak rozlišujeme
počiatočný a koncový vrchol pri hranách grafu, hovoríme, že hrany sú
orientované. Ak sú počiatočný a koncový vrchol hrany totožné, hovoríme, že
táto hrana je slučka. Hrana vedie medzi počiatočným a koncovým vrcholom. Ka
medzi niektorou dvojicou vrcholov vedie viacero hrán, hovoríme, že tieto
hrany sú násobné.
V rámci Teórie grafov sa skúmajú grafy z rôznych pohľadov a riešia sa na 
nich najrôznejšie úlohy. My sa budeme zaoberať grafmi bez slučiek,
orientovaných a násobných hrán.
Grafy boli na základe vybraných vlastností roztriedené do viacerých tried.
Roztriedenie grafov medzi jednotlivé triedy umožnilo 
pozrieť sa na konkrétne prípady mnohých problémov pre ktoré nebolo známe 
všeobecné riešenie a dosiahnuť aspoň čiastočné výsledky. Taktiež to umožnilo 
optimalizovať všeobecné algoritmy a dosiahnuť na špecifickejších vstupoch 
rádovo lepšiu časovú náročnosť výpočtu.


Medzi významné problémy týkajúce sa grafov patria problémy ich
konštrukcie a prehľadávania. V súčasntosti sa problému konštrukcie grafov
venuje takmer výlučne teória grafových gramatík. Tento prístup používa
podobné prostriedky a postupy ako známejšia teória formálnych jazykov a
automatov s tým rozdielom, že grafové gramatiky sa týkajú grafov a sú teda o
dosť zložitejšie. Napríklad pojem bezkontextovej grafovej gramatiky má
viacero rozdielnych prirodzených interpretácií. V tejto teórii sa vyvinulo
viacero prístupov a skúmaných modelov.
Ďalším spomínaným problémom týkajúcim sa grafov je problém ich prehľadávania.
V rámci teórie prehľadávania grafov sa skúmajú viaceré modely. Väčšina z
nich má však spoločný základ. Týmto základom je entita nazývaná agent
prípadne robot, automat atď., ktorá sa pohybuje po vrcholoch daného grafu 
a jeho úlohou
je tento graf prehľať. Jednotlivé modely sa môžu líšiť vo viacerých 
vlastnostiach.
Medzi najdôležitejšie z nich patrí vlastnosti agenta i spôsob jeho pohybu 
po grafe, vlastnosti grafu a definícia pojmu prehľadať. Prehľadanie grafu môže
znamenať navštíviť každý vrchol grafu, ale tiež prejsť po každej hrane.
Taktiež je dôležité, aké informácie získa agent o svojom okolí pri návšteve
niektorého vrchola. Môže napríklad vidieť všetky vrcholy spojené hranou alebo
len lokálne čísla portov.
Naša práca skúma možnosti konštrukcie grafov pomocou mobilných agentov.
Spája teda obe bližšie spomínané teórie. Teórie grafových gramatík sa týka v
tom, že sa tiež zaoberá konštrukcio grafov. S prehľadávaním grafu pomocou
agentov má spoločného viac. Ide totiž o akýsi duálny problém. Pri
prehľadávaní ide často o získanie akejsi mapy neznámeho grafu a v našej
práci sa naopak zaoberáme konštrukciou grafu pri znalosti mapy. V prvom
prípade teda máme graf a chceme mapu, kdežto v druhom máme mapu a chceme
graf.
Keďže problém konštrukcie grafov pomocou mobilných agentov nebol dosiaľ
skúmaný, zvolili sme si na skúmenie jednoduchý model.
Na ňom sme skúmali možnosti konštrukcie grafov z jednotlivých tried. Zvlášť
sme sa venovali skúmaniu konštrukcie hyperkocky a tried grafov $P_{n} \times
P_{m}$ i $C_{n} \times C_{m}$. Výsledky na týchto triedach nájdete v
hlavných kapitolách práce.
Náš model sa skladá z
agenta nachádzajúceho sa v počiatočnom vrchole grafu, ktorý ide konštruovať.
Tento agent môže vždy vykonať jednu z troch možných operácií. Agent pozná
lokálne značenie koncov hrán - portov vo vrchole, v ktorom sa nachádza.
Toto značenie je dané poradím ich vzniku a agent nemá možnosť ho meniť.
Operácie, ktoré môže agent vykonať, sú: otvorenie portu s danou značkou,
prejdenie po hrane s lokálnou značkou portu $k$ i vytvorenie novej hrany do
dosiaľ neskonštruovaného vrchola; v ktorom sa agent vzápätí ocitne. K
vlastnostiam modelu tiež patrí, že v momente, keď sú otvorené dva porty s
rovnakou značkou, vznikne medzi nimi hrana.
Krok agenta sme si definovali ako zmenu vrchola, v ktorom sa agent nachádza.
Skúmali sme aký minimálny počet krokov agenta je potrebný na konštrukciu
daného grafu pri rôzne veľkých množinách značiek, ktoré má agent k
dispozícii.

