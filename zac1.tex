%\cleardoublepage
\phantomsection
\addcontentsline{toc}{chapter}{Úvod}
\chapter*{Základný model}\label{chap:intro}

\section{Definícia}
\begin{defin}
Model sa skladá z grafu a agenta. Graf má nemeniteľné lokálne číslovanie koncov hrán vo
vrchole, podľa poradia vzniku. Koniec hrany, ktorý vznikol ako prvý, má
číslo 1. Prvý koniec vznikne spolu s vrcholom, zvyšné otvorením portu.
Počiatočný graf obsahuje práve jeden izolovaný vrchol a nič viac. Akonáhle
sú otvorené dva porty s rovnakými značkami, vznikne medzi nimi nová hrana a
tieto porty tým zaniknú.
\\
Agent môže vo vrchole, v ktorom sa nachádza, vykonávať tieto operácie: pohnúť sa po hrane do susedného
vrcholu, otvoriť port so značkou, vytvoriť hranu do nového vrchola v ktorom sa následne
ocitne. Agent rozlišuje hrany podľa lokálneho číslovania ich koncov vo
vrchole, kde sa práve nachádza. Agent vie ktorý koniec patrí hrane po ktorej
prišiel do vrcholu. Agent začína vo vrchole počiatočného grafu.
\\
Agent pri vykonávaní algoritmu konštrukcie grafu nemá obmedzenú výpočtovú
silu ani pamäť. Množina značiek portov je konečná a vopred daná. Efektivita
algoritmu konštrukcie daného grafu sa pri tomto modely meria počtom pohybov
agenta (tj. sumou počtu návštev cez všetky vrcholy) pri danej množine značiek
portov, tento počet sa snažíme minimalizovať.
\end{defin}
