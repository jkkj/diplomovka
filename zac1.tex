%\cleardoublepage
\phantomsection
\addcontentsline{toc}{chapter}{Úvod}
\chapter*{Model}\label{chap:intro}

\section{Definícia}
\begin{defin}
Model sa skladá z grafu a agenta. Graf má nemeniteľné lokálne číslovanie koncov hrán vo
vrchole, podľa poradia vzniku. Koniec hrany, ktorý vznikol ako prvý, má
číslo 1. Prvý koniec vznikne spolu s vrcholom, zvyšné otvorením portu.
Počiatočný graf obsahuje práve jeden izolovaný vrchol a nič viac. Akonáhle
sú otvorené dva porty s rovnakými značkami, vznikne medzi nimi nová hrana a
tieto porty tým zaniknú.
\\
Agent môže vo vrchole, v ktorom sa nachádza, vykonávať tieto operácie: pohnúť sa po hrane do susedného
vrcholu, otvoriť port so značkou, vytvoriť hranu do nového vrchola v ktorom sa následne
ocitne. Agent rozlišuje hrany podľa lokálneho číslovania ich koncov vo
vrchole, kde sa práve nachádza. Agent vie ktorý koniec patrí hrane po ktorej
prišiel do vrcholu. Agent začína vo vrchole počiatočného grafu.
\\
Agent pri vykonávaní algoritmu konštrukcie grafu nemá obmedzenú výpočtovú
silu ani pamäť. Množina značiek portov je konečná a vopred daná. Efektivita
algoritmu konštrukcie daného grafu sa pri tomto modely meria počtom pohybov
agenta (tj. sumou počtu návštev cez všetky vrcholy) pri danej množine značiek
portov, tento počet sa snažíme minimalizovať.
\end{defin}

\begin{ozn}
Označenie operácií agenta:
- nová hrana = NH
- otvoriť port (s číslom n) = OP(n)
- prejsť sa po hrane s lokálnym číslom konca k = Sk
\end{ozn}

\begin{ozn}
Agent začína v grafe, ktorý sme už nazvali počiatočný graf - v našom prípade
obsahuje jeden izolovaný vrchol. Graf, ktorý má agent skonštruovať, budeme
volať cieľový, konečný alebo konštrupvaný. Graf existujúci v danom kroku
konštrukcie v modely budeme volať aktuálny graf alebo len graf.
\end{ozn}

\begin{pozn}
Pri práci s neorientovanými grafmi používame pojem "hrana vedie z vrchola
$v$
do vrchola $u$" len na rozlíšenie týchto dvoch vrcholov od seba.
\end{pozn}

\begin{ozn}
Nad prvkami grafu, konštruovaného jedným agentom
  (sú to najmä vrcholy, hrany i porty), existuje prirodzené úplné
usporiadanie podľa poradia ich vzniku, podľa ktorého môžeme o jednom prvku
vyhlásiť, či je starší, mladší alebo rovnako starý ako iný prvok. Platí, že
hrana je rovnako stará ako jej druhý port. Všeobecne prvky vzniknuté jednou
operáciou sú rovnako staré.
\end{ozn}

\begin{ozn}
Krokom agenta budeme nazývať vykonanie operácie, pri ktorej sa zmení vrchol,
v ktorom sa agent nachádza.
\end{ozn}

\begin{ozn}
Portová hrana je hrana, ktorá vznikla operáciou OPk, kde k je značka portu,
ktorého otvorením hrana vznikla.
\end{ozn}
\begin{ozn}
Kostrová hrana je hrana, ktorá vznikla operáciou NH.
\end{ozn}
\begin{pozn}
Ako neskôr ukážeme - "kostrové hrany" tvoria kostru grafu.
\end{pozn}

\begin{ozn}
Medzi otovrením prvého a druhého portu portovej hrany $h$, prejde agent po slede
vrcholov. Najmenší cyklus z tohto uzavretého sledu obsahujúci
 hranu $h$ budeme označovať tvorivý cyklus hrany $h$. Hrany tohto cyklu bez
hrany $h$ tvoria konštrukčnú cestu hrany $h$.
\end{ozn}

\begin{ozn}
Postupnosť návštev vrcholov a hrán v poradí v akom ich agent vykonal počas
konštrukcie cieľového grafu budeme volať tvorivý sled grafu.
\end{ozn}
\section{Základné vlastnosti}
\begin{lem}
Agent počas pohybu v grafe musí ísť po hranách grafu. 
Tieto po momente keď po nich prešiel, už v grafe trvale existujú.
\end{lem}

\begin{proof}
Žiadna operácia agentovi nedovoľuje priamo prejsť medzi dvoma vrcholmi, ktoré
nespája hrana. Môže však vytvoriť kostrové hrany a potom otvorením portu
vznikne hrana, ktorá mu umožní dostať sa do vrcholov grafe, ktoré vytvoril
predtým. Neexistuje spôsob ako v našom modely odstrániť hranu z grafu.
\end{proof}

\begin{lem}
Ak agent odíde z vrcholu v ktorom je súčet počtov incidentných hrán a
otvorených portov nižší, ako je počet hrán s ktorými má byť vrchol
incidentný v cieľovom grafe, tak sa doň počas konštrukcie musí agent neskôr
ešte vrátiť.
\end{lem}

\begin{proof}
Neexistuje postupnosť operácií agenta, pomocou ktorej môže agent vytvoriť
hranu do
existujúceho vrchola bez jeho návštevy, ak v ňom nie je otvorený port pre
túto hranu. Operácia NH vytvorí z vrchola v ktorom sa nachádza do nového
vrcholu, ktorý predtým v grafe neexistoval a operáciou OPk môže agent
vytvoriť hranu z vrchola v ktorom sa nachádza do vrchola v ktorom je
otvorený port so značkou k. Teda ak sa
v danom vrchole agent nenachádza a nemá v ňom otvorený port pre hranu,
ktorú ide skonštruovať, nemôže to urobiť bez návštevy tohto vrchola.
\end{proof}

\begin{lem}
Hrany vytvorené operáciou "nová hrana" tvoria kostru grafu.
\end{lem}
\begin{proof}
Okrem inicializačného vrcholu, všetky vrcholy vzniknú operáciou "nová hrana"
(NH). Každý vrchol okrem inicializačného je spojený hranou, s lokálnym číslom
konca hrany
1, s nejakým starším vrcholom (vrchol, ktorý vznikol skôr). Po týchto hranách 
sa dá
z každého vrchola grafu dostať do inicializačného, teda podgraf, 
tvorený hranami
pochádzajúcimi z oprácií NH, je súvislý. 
Počet týchto hrán je zároveň o jeden menší ako počet
vrcholov, teda ide o kostru.
\end{proof}

\begin{lem}
\label{uzavrety_sled}
Hrany, ktoré agent prejde od otvorenia prvého portu hrany po otvorenie druhého,
tvoria v grafe s novovzniknutou hranou uzavretý sled.
\end{lem}
\begin{proof}
Je to dané operáciami, ktoré môže agent vykonať. Agent totiž prechádza len
po hranách v grafe. Z vrchola ide do incidentnej hrany a z hrany do
incidentného vrchola. Pričom žiadna hrana nemôže zaniknúť, teda medzi otvorením
dvoch portov prejde agent po nejakom slede v grafe. Novovzniknutá hrana
spája koncový a začiatočný bod tohot sledu; tvorí s ním teda uzavretý sled.
\end{proof}
\begin{pozn}
Zo sledu z vety \ref{uzavrety_sled} vieme vybrať kružnicu.
\end{pozn}
\begin{lem}
\label{minimum}
Minimálny počet hrán, ktoré agent potrebuje prejsť pri konštrukcii 
hrany pomocou
portov je rovný alebo väčší ako dĺžka najmenšej kružnice, na ktorej sa táto
hrana v konštruovanom grafe nachádza.
\end{lem}

\begin{proof}
Nájdeme menší takýto sled ako je najkratšia kružnica. Tento tvorí s
novovzniknutou hranou uzavretý sled. Z neho vieme vybrať
kružnicu, čo je spor s minimálnosťou najkratšej kružnice. Keďže neexistuje v
našom modely možnosť, aby hrana zanikla, musí sa táto "nová" najkratšia
kružnica nachádzať aj v konštruovanom grafe.
\end{proof}

\begin{lem}
Ak má agent k dispozícii dostatočný počet značiek portov, 
tak sa problém efektívnej konštrukcie grafu
redukuje na hľadanie najkratšieho sledu v ktorom sú všetky vrcholy. Pri
hamiltonovskom grafe ide o hamiltonovskú kružnicu.
\end{lem}
\begin{proof}
Agent skonštruuje graf takto: prechádza vrcholmi, podľa najkratšieho sledu
obsahujúceho všetky vrcholy, pričom mu sačí jedna návšteva každého vrcholu,
 v ktorom
otvorí potrebný počet portov príslušných značiek. Ak má v slede prejsť do
vrcholu, ktorý ešte neexistuje, použije operáciu NH.
\end{proof}

\begin{pozn}
Aj v tomto prípade je možná ďalšia optimalizácia a to na počet použitých
značiek pre porty. Celkovo to funguje takto: agent otvorí port so značkou k, vytvorí pár
kostrových hrán a následne v nejakom vrchole otvorí druhý port s rovnakou
značkou, čím sa mu táto uvoľní. V každom vrchole, ktorý medzitým vytvorí,
musí otvoriť porty, pre všetky hrany v konštruovanom grafe, s ktorými je v
ňom tento vrchol incidentný. Pri tom nemôže použiť značku k. Vo všeobecnosti
teda platí, že čím kratšie sú porty otvorené, tým menší je celkový počet
potrebných značiek.
\end{pozn}

\begin{lem}
Vrchol v grafe musí agent navštíviť minimálne $\frac{deg(v)}{|Z| * 2} $ krát,
kde Z je množina značiek pre porty, ktorú ma agent k dispozícii.
\end{lem}
\begin{proof}
Pri každej návšteve môže agent otovrením portov so všetkými značkami 
vytvoriť nové hrany do vrcholov, v ktorých sú otovrené príslušné porty 
a následne otvoriť toľko
portov s navzájom rozdielnymi značkami, koľko má k dispozícii značiek
(dvojica portov rovnakej značky v jednom vrchole by vytvorila slučku a v
našom modely uvažujeme hrany bez slučiek).
\end{proof}
\begin{pozn}
Ako ukážeme neskôr, pri obmedzenom množstve značiek portov môže byť
výhodnejšie neminúť všetky značky pre porty v jednom vrchole, aj za cenu
opakovanej návštevy. Výhodnejšie v tomto prípade znamená, že agent pri danej
konštrukcii vykoná menej krokov.
\end{pozn}


\subsection{Jedna značka pre porty}

\begin{lem}
Najmenej krokov agent na konštrukciu konkrétnej hrany medzi už existujúcimi
vrcholmi použije, ak pôjde do najbližšieho z nich, otvorí v ňom port a
následne najkratšou cestou po už existujúcich hranách prejde do druhého
vrcholu a otvorí v ňom port.
\end{lem}
\begin{proof}
Agent má k dispozícii len jednu značku pre porty. Teda počas konštrukcie
hrany medzi už existujúcimi vrcholmi, žiadne nové hrany v grafe,
 ktoré by ležali na ceste medzi koncami tejto hrany nevznikn.
 Jediné nové hrany môžu vzniknúť operáciou NH, ale
tieto by viedli do "slepej uličky" (komponentu grafu obsahujúceho len nové
vrcholy a spojeného mostom so zvyškom grafu), z ktorej by sa musel agent vracať a je
zrejmé, že ak ich konštrukciu vynechá, z pohľadu najmenšieho počtu ťahov na
konštrukciu tejto portovej hrany nič nepokazí. V neorientovanom grafe
pri jednej značke pre port je najkratšia cesta od otvorenia prvého po
otvorenie druhého portu hrany rovnaká bez ohľadu na to, 
v ktorom vrchole agent otvorí prvý port hrany.
Jediné, čo sa mení, je počet krokov po otvorenie prvého z portov.
Aby bol tento a teda aj celkový počet krokov čo najkratší, pôjde agent do
najbližšieho z vrcholov konštruovanej portovej hrany.
\end{proof}

\begin{lem}
Agent, ktorý má k dispozícii len jednu značku pre porty, 
musí prejsť pri konštrukcii grafu aspoň k krokov, kde 
$k =min_{i=1}^{y}\{\sum_{e \in G(E); e \notin G(K_{i})} j(e)-1  \}$, 
kde K je kostra grafu G; y je počet navzájom rôznych kostier garfu G; j(e) je najmenšia kružnica v
grafe obsahujúca hranu e a $K_{i}$ je i-ta kostra, pričom $K_{i} \equiv K_{j}
\iff i \equiv j$
\end{lem}
\begin{proof}
Podľa vety \ref{minimum} musí agent prejsť pri konštrukcii hrany aspoň dĺžku
minimálnej kružnice bez jednej - bez hrany ktorú tým konštruuje. Keďže má k
dispozícii len jednu značku pre porty, ďalšiu hranu môže začať konštruovať
až keď dokončil predošlú. Pri každej hrane prejde teda aspoň dĺžku
najmenšej kružnice, v ktorej sa táto hrana nachádza mínus jedna. Dokopy
prejde agent ich súčet.
\end{proof}



\begin{lem}
Po použití operácie NH, ktorou vznikne hrana h, sa agent už nedostane do
už existujúceho vrchola bez následného použitia
operácie OP, ktorá vytvorí hranu do vrcholov starších ako h,
alebo bez opätovného prejdenia po hrane h.
\end{lem}
\begin{proof}
Vrchol v, ktorý vznikne operáciou NH je so zvyškom grafu spojený len hranou
h, ktorá tvorí most, do vzniku hrany operáciou OP, medzi komponentmi na oboch
stranách tohto mostu. Teda jediný spôsob ako prejsť medzi týmito dvomi
časťami grafu je po moste, ktorý ich spája - po hrane h.
\end{proof}

\begin{lem}
Existuje graf, pri ktorého konštrukcii na najmenej krokov, agent
skonštruuje kostrové hrany, ktoré s konštruovanou portovou hranou netvoria
tvoriaci cyklus.
\end{lem}
\begin{proof}
Napríklad cyklus, kde do každého vrchola pridáme hranu do vrchola stupňa
jedna. Je zrejmé, že pri konšrukcii týchto vrcholov stupňa jedna, agent
urébí operácie NH a S1, lebo nemôže ostať vo vrchole ak ešte nedokončil graf
a iná hrana z neho nevedie. Tiež je zrejmé, že takýto graf má práve jednu
portovú hranu a to ľubovoľnú hranu cyklu. Pri jej konštrukcii prejde agent
cez všetky vrcholy v cykle a ak by neskonštruoval príslušný vrchol stupňa
jedna incidentný s vrcholom cyklu, pri jeho prvej návšteve, musel by sa doň
vrátiť a teda vykonať viac krokov. Agent treda otvorí port a skonštruuje
vrcholy cyklu, pričom v každom skonštuuje aj príslušný vrchol stupňa jedna s
ktorým je tento incidentný.
\end{proof}


\subsection{Dve a viac značiek pre porty}
Ušetriť (agent vykoná menej krokov) oproti situácii, 
keď agent môže použiť len jednu značku pre porty
možno v prípade že má k dispozícii viaceron značiek pre porty vtedy,
 ak konštruovaný graf obsahuje cykly, kotré majú spoločné hrany.
Vtedy, ak agent pri konštrukcii niektorej hrany, 
prechádza vrcholom s ktorým bude incidentná niektorá
portová
hrana, ktorú bude konštruovať po tejto,
 v ňom otvorí port. Následne sa do tohto vrchola už nemusí pri konštrukcii
tejto nasledujúcej portovej hrany vracať otvoriť port, ak existuje
kratšia cesta do druhého koncového vrchola tejto hrany. V metrike
vzdialeností, kde sa počíta počet hrán najkratšej cesty medzi dvomi vrcholmi
ako ich vzdialenosť platí trojuholníková nerovnosť. Z tohto dôvodu agent
nevykoná viac krokov pri najlacnejšej (na najmenej krokov) 
konštrukcii portovej hrany ak jeden port tejto hrany otvorí skôr oproti
konštrukcii kde ho otvorí neskôr, pričom sa kroky rátajú až po dokončení
predošlej portovej hrany. Krok, vykonaný agentom v čase, keď má tento
otvorených viacero porto, sa do celkového súčtu počíta len raz.
Ako neskôr ukážeme, nie vždy spôsobí snaha skonštruovať jednu hranu na čo
nejmenej krokov zlepšenie celkového počtu krokov agenta počas konštrukcie
cieľového grafu.
\chapter{Všeobecná konštrukcia}\label{chap:intro}
